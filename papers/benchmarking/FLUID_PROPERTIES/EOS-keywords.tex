\subsection{Fluid specification}\label{sec-fluid-name}
The new fluid property functions are working for specific substances. So, it is necessary to specify the desired fluid in the \texttt{*.MFP}-file. This has the advantage, that there is no need to know or to look-up fluid properties like initial density or viscosity for a specific pressure or temperature condition. The subkeyword \texttt{\$FLUID$\_\,$NAME} identifies the substance. Since GeoSys version 4.10.02, four substances are defined: \co2, \h2o, \ch4 and \n2. By defining the fluid name, all necessary coefficients, parameters and critical values are called by the source code to solve the EOS and the fluid property functions. 
\paragraph{Example:}
In \texttt{MFP}-file, write:
\begin{verbatim}
#FLUID_PROPERTIES
 $FLUID_TYPE
  LIQUID
 $FLUID_NAME
  CARBON_DIOXIDE
\end{verbatim}
The first letter of the fluid name identifies the substance, so 'C' would be sufficient define the substance clearly as \co2. Table \ref{tab-fluid_name_spec} shows all available substances and their identifiers. 


\begin{table}[ht]
%\centering
\caption{Fluid substances and their identifiers.}
\begin{center}
\begin{tabular}{cl}
\toprule
Identifier & Substance \\
\midrule
\texttt{W} & Water \\
\texttt{C} & Carbon dioxide \\
\texttt{M} & Methane \\
\texttt{N} & Nitrogen\\
\bottomrule
\end{tabular}
\end{center}
\label{tab-fluid_name_spec}
\end{table}


\subsection{Density model specification}

With these specified substances, it is possible to choose between different equations of state, which are described in detail in section \ref{eos-density}. In addition to former ones, a selection of new density models is available since GeoSys version 4.10.00. The refering density model identifiers are shown in table \ref{tab-density-models}. These density models determine the fluid density depending on pressure and temperature. If multiple phases or components are defined, an optional argument name can be specified behind the density model identifier.

\paragraph{Example:}
(\texttt{MFP}-file)
\begin{verbatim}
 $FLUID_NAME
  CARBON_DIOXIDE
 $DENSITY
  12 PRESSURE_W TEMPERATURE1
\end{verbatim}

This example sets up density model 12 (PREOS), where \texttt{PRESSURE\_W} and \texttt{TEMPERATURE1} serve as property function arguments. If the optional argument specification is omitted, the default argument names are \texttt{PRESSURE1} and \texttt{TEMPERATURE1}. So, especially for multiphase-flow examples, the declaration of arguments can be important. For isothermal problems, it is possible to define a reference temperature behind the density model identifier.

\paragraph{Example:}
(\texttt{MFP}-file)
\begin{verbatim}
 $FLUID_NAME
  CARBON_DIOXIDE
 $DENSITY
  13 400 PRESSURE2
\end{verbatim}

In this case, the temperature argument for the equation of state is a constant value of $\unit[400]{K}$, regardless if heat transport considered or not.

\begin{table}[ht]
%\centering
\caption{Density model identifiers}
\begin{center}
\begin{tabular}{cl}
\toprule
Identifier & Density model \\
\midrule
\texttt{10} & Lookup-table\\
\texttt{11} & \textsc{Redlich\&Kwong} \\
\texttt{12} & \textsc{Peng\&Robinson} \\
\texttt{13} & \textsc{Helmholtz} free energy\\
\bottomrule
\end{tabular}
\end{center}
\label{tab-density-models}
\end{table}

\subsection{Viscosity model specification}

For viscosity, several correlation functions were presented by different authors (see section \ref{sec-viscosity}), but there is only one new viscosity model identified by no. 9. When viscosity model is 9, the right correlation function is chosen automatically by the fluid name (see section \ref{sec-fluid-name}). If no fluid name is specified, the default substance will be carbon dioxide. For viscosity model 9, the same arguments as for density models 11 to 13 can be given.

\paragraph{Example:}
(\texttt{MFP}-file)
\begin{verbatim}
#FLUID_PROPERTIES
 $FLUID_TYPE
  LIQUID
 $FLUID_NAME
  CARBON_DIOXIDE
 $VISCOSITY
  9 400 PRESSURE2
\end{verbatim}
%$
\subsection{Thermal conductivity model specification}

The thermal conductivity correleations depending on temperature and pressure can be switched on by identifier '3'.

\paragraph{Example:}
(\texttt{MFP}-file)
\begin{verbatim}
 $HEAT_CONDUCTIVITY
  3 PRESSURE2
\end{verbatim}
%$
The \textbf{heat conductivity model 3} handles the same arguments as \textbf{viscosity model 9}. in section \ref{sec-heat-conductivity}, the correlation functions for all implemented substances are described and referenced.


\subsection{Compressibility model specification}

Since GeoSys version 4.10.02, a fluid can be considered as compressible or incompressible due to temperature and pressure\footnote{only in global pressure-saturation formulation}. The compressibility of a fluid depends on the slope of the density function surface (see \eqref{eq_gov_specific} on page \pageref{eq_gov_specific}). This slope can be determined by

\begin{itemize}
\item a constant value
\item a lookup-table
\item the difference quotient
\item the analytical derivation of the equation of state
\end{itemize}

or can be set to zero to treat the fluid as incompressible. In GeoSys version 4.10.02, only the constant value and the difference quotient alternatives are available. This difference quotient is a simple and fast way to approximate the slope of the density function:

\begin{equation}
\frac{\partial \rho_\alpha}{\partial p_\alpha} \approx \frac{\Delta \rho_\alpha}{\Delta p_\alpha} = \frac{\rho_\alpha\left(p_\alpha+\frac{\Delta p_\alpha}{2},T\right)-\rho_\alpha\left(p_\alpha-\frac{\Delta p_\alpha}{2},T\right)}{\Delta p_\alpha}.
\label{eq-diff-quotient}
\end{equation}


A comparable equation determines the slope of the density function in terms of temperature. The compressibility model can be defined for each individual phase. The new subkeyword \texttt{\$COMPRESSIBILITY} requires two lines of arguments: The first line defines the fluids compressibility due to pressure, the second one the compressibility due to temperature. Each argument line has to start with the compressibility model identifier. The meaning of the second argument depends on the chosen identifier (see Table \ref{tab-compressibility-identifiers}). The following example defines carbon dioxide as compressible due to pressure changes using a numerical approximation of its equation of state (model 3, first line); the pressure difference $\Delta p$, which is used to detemine the difference quotient (see \eqref{eq-diff-quotient}) is set to $\unit[1]{Pa}$. Furthermore, the example defines the fluid as incompressible due to temperature changes (model 0, second line).

\paragraph{Example:}
(\texttt{MFP}-file)
\begin{verbatim}
 $FLUID_NAME
  CO2
 $COMPRESSIBILITY
  3 1
  0
\end{verbatim}

\begin{table}[ht]
%\centering
\caption{Compressibility model identifiers}
\begin{center}
\begin{tabular}{cll}
\toprule
Identifier & Argument & Compressibility model \\
\midrule
\texttt{0} & - & incompressible\\
\texttt{1} & $\frac{\Delta \rho}{\Delta p}$ or $\frac{\Delta \rho}{\Delta T}$ & constant slope \\
\texttt{2} & - & lookup-table \\
\texttt{3} & $\Delta p$ (eq. \ref{eq-diff-quotient}) or $\Delta T$ & difference quotient\\
\texttt{4} & - & analytical derivation\\
\bottomrule
\end{tabular}
\end{center}
\label{tab-compressibility-identifiers}
\end{table}

\subsection{Example file}

The following listing of an \texttt{MFP}-file defines a two-phase or a two-component scenario with \h2o and \co2 as involved fluids.\\[1.5ex] 

Water is incompressible due to pressure changes, but its density depends on temperature. The chosen equation of state is the fundamental \textsc{Helholtz} free energy equation, which is the only EOS which returns accurate density values for water, but takes long computing time. The viscosity and the thermal conductivity of water is determined by the IAPWS formulation for scientific use \cite{IAPWS:08a}, \cite{IAPWS:08b}. No pressure or temperature argument names are defined for water, so the default \texttt{PRESSURE1} and \texttt{TEMPERATURE1} variables are used for fluid no. 1.\\[1.5ex] 

The example file considers carbon dioxide as compressible due to pressure and temperature, where the temperature depending density changes are approximated by a constant slope ($\unit[2.5]{kg\cdot m^{-3}\cdot T^{-1}}$). \co2 is a much less complex than water, so a simple PREOS is precise enough for density determination. For viscosity and thermal conductivity, the formulations of \textsc{Fenghour} et al. \cite{FenWakVes:98} are chosen. For this second fluid, the argument name \texttt{PRESSURE2} has to be declared.

\paragraph{Example:}
(\texttt{MFP}-file)
\begin{verbatim}
#FLUID_PROPERTIES
 $FLUID_TYPE
  LIQUID
 $FLUID_NAME
  WATER
 $COMPRESSIBILITY
  0
  3 1
 $DENSITY
  13    
 $VISCOSITY
  9
 $HEAT_CONDUCTIVITY
  3
#FLUID_PROPERTIES
 $FLUID_TYPE
  GAS
 $FLUID_NAME
  CO2
 $COMPRESSIBILITY
  3 1
  1 2.5
 $DENSITY
  12 PRESSURE2
 $VISCOSITY
  9  PRESSURE2
 $HEAT_CONDUCTIVITY
  3 PRESSURE2
#STOP
\end{verbatim}

\subsection{Fluid property output}

Fluid properties can be written in the output-file using the subkeyword \texttt{\$MFP$\_\,$VALUES} in \texttt{*.OUT}-file. This was only possible for one single phase or component. Since GeoSys version 4.10.02, it is possible to define the number of the phase, whose properties shall be given out. This can be done by writing the phase number directly behind the respective fluid property.

\paragraph{Example:}
(\texttt{OUT}-file)
\begin{verbatim}
 $MFP_VALUES
  DENSITY1
  HEAT_CONDUCTIVITY1
  DENSITY2
  VISCOSITY2
\end{verbatim}
%$

This example writes density and thermal conductivity of the first, and density and viscosity of the second defined fluid into the respective output-file.

