%-------------------------------------------------------------------------
\subsection*{THMC Proceses}

Many applied recent problems in geoscience and geoengineering, such as chemo-toxic and radio-nuclear waste deposition, geothermal energy utilization, carbon dioxide sequestration, require profound knowledge about complex interactions between thermal, hydraulic, mechanical and chemical processes in the subsurface. As a direct experimental investigation is often not possible, numerical simulation is a common approach to analyse processes and develop scenarios.

The methodology this book is organized into the following four parts: chemico-physical basics, continuum mechanics, numerical methods and computational aspects for the simulation of THMC processes in porous media.

\subsubsection*{Chemico-physical basics - conceptual modeling}
In the context of THMC analysis we have to consider flow, mass and heat transport, deformation as well as chemical processes simultaneously. Many of these processes are strongly coupled, e.g. flow and transport, flow and deformation, transport and reaction, and this necessitate a fully coupled analysis. Other systems are rather weakly linked, e.g. transport and deformation. Depending on the specific thermo-dynamical situation, the conceptual modelling needs to define the degree of coupling and, therefore, the adequate continuum-mechanical model (section 2).

\subsubsection*{Continuum mechanics}
Continuum mechanics is the conceptual basis for the analysis of THMC processes. As we have to deal with porous media (representing soils, host rock, aquifers) we have to consider a multi-phase system consisting of solid (rock matrix, soil grains), several fluid (water, oil, gaseous), mineral and biological phases. The mathematical framework for the description of THMC processes is the balance equations for mass, momentum, and energy. Some of these equations need to be formulated for individual phases (e.g. for multi-phase flow). If local thermodynamic equilibrium can be assumed (e.g. for thermal processes) phase-related formulations can be superposed and a general form of balance equation for the porous medium can be derived.

\subsubsection*{Numerical methods}
The mathematical formulation of THMC problems in porous media results in a set of coupled non-linear partial differential equations (PDEs), which usually needs to be solved numerically. Analytical solutions only exist for exceedingly simplified problems. For practical problems, e.g. 3-D geometries with irregular boundary conditions, approximation methods need to be applied. Several numerical methods exist for the solution of THMC such as the finite element method (FEM) or the finite volume method (FVM). FEM is based on global variation principles whereas FVM is founded on local balance considerations. We prefer FEM for the numerical solution of THMC problems. As a result of the FEM procedure a system of algebraic equations is derived. Depending on the degree of coupling, monolithic (for strong coupling) or staggered (partitioned) schemes are used. In particular for monolithic algorithms large equation systems have to be solved, therefore, efficient computational methods are necessary (section 4).

\subsubsection*{Computational aspects}
The numerical analysis of THMC problems is computational extremely expensive and the applicability of existing codes is still limited to simplified problems. Therefore, the improvement of computational schemes is very important for the solution of practical problems. We use two methods from compute science to this purpose: Object-orientation and parallel computation. Object-orientation is an informatics method for the design of the software architecture. Object-oriented programming is using so called "classes". A class is a template for specific data objects (e.g. boundary conditions, material properties, matrix structures) and corresponding data processing methods. Instances (i.e. copies) of classes are used for similar data objects. It is note worthy that, despite complexity, we succeeded to implement a class concept for THMC problems which is completely independent on the specific physico-chemical processes. Furthermore, parallelization techniques are used to improve the computational performance for expensive numerical THMC simulations on modern parallel computer platforms. Today's supercomputers contain several thousand CPUs. In order to use these high-performance-computers codes need to be parallelized. Basically we implemented parallel routines for element level assembly and for parallel solution of the resulting equation systems. MPI instructions are used for inter-processor communication.

\subsubsection*{Application examples}
We present two examples of THMC analysis for geothermal reservoir engineering and for nuclear waste deposition. THM analysis was conducted for the geothermal reservoir at Urach Spa which is located in southern Germany. Within the DECOVALEX international benchmarking project we performed THM and THC studies for nuclear waste deposition scenarios. In general, two different concepts exist for nuclear waste repositories. The first is using engineered barriers, e.g. consisting of bentonite buffers (FEBEX type repository). The second is based on the capillary barrier effect which is applicable in very dry areas (Yucca Mountain type repository).
