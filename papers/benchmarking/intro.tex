\chapter{Introduction}

%-------------------------------------------------------------------------
\subsection*{Scopus}

The intention of the Developer Benchmark Book (DBB) is twofold:
\begin{itemize}
  \item DDB provides a collection of test cases which are used for benchmarking the GeoSys code development.
  \item We recommend the benchmark collection for users as a starting point for their own GeoSys applications.
\end{itemize}

%-------------------------------------------------------------------------
\subsection*{RockFlow, GeoSys, OpenTHMC, a historical note ...}

Looking back to a more than 20 years lasting scientific project as ''RockFlow'' is amazing ... and at the same time completely impossible to be comprehensive and to be fair ...

\subsubsection*{RockFlow-1}

Somewhen in the mid eighties Dr. Liedtke with the Federal Institute of Geosciences (BGR) was asking Prof. Zielke (Institute of Hydromechanics, University of Hannover) whether the development of a simulation programm for fractured rock is possible. (Never ask a scientist about impossible things). The idea was born (including some funding from the BGR): the development of a computer code based on multi-dimensional FEM. The first name was DURST, which for Germans is not really a good choice (because it means "thirsty"). Later it was renamed into Rockflow: Flow and associated processes in rock. The pioneering work of RockFlow-1 was done in the following four doctoral dissertations \cite{Kro:1990}, \cite{Wol:1991}, \cite{Hel:1993}, \cite{Sha:1994}.

\subsubsection*{RockFlow-2}

The next stages in the early nineties was related to couple the individual RF-1 modules and improve the computational efficiency, e.g. by introducing iterative equation solver. RockFlow-2 was now used in several application projects as waste deposition and geothermal energy (\cite{Leg:1995},\cite{Kol:1996}). A "market" for RockFlow in Applied Geoscience was open.
From this time the most cited RockFlow paper so far \cite{KolRatDieZie:98} originated (more than 70 times, which is not so bad for a modeling paper).

\subsubsection*{RockFlow-3}

It turned out that the coupling of the different RF modules needed a new code structure.
Moreover for the use of grid-adaptive methods dynamic data structures have been necessary.
Consequently, in the late nineties a complete re-organization of RF was started. C experience began ...
\cite{MSR:1996}, \cite{Bar:1997}.
Major research topics of the RF group had been multi-phase flow \cite{Tho:01}, grid adaptation \cite{Kai:01}, reactive transport \cite{Hab:01}, and deformation processes
\cite{Koh:2006}. Beside the numerical parts geometric modeling and meshing methods became more and more important \cite{Rot:01}, \cite{Moe:2004}.

\subsubsection*{GeoSys/RockFlow-4}

Tuebingen:
Due to the increasing functionality, the RF code became more and more sophisticated and difficult to handle.
Consequently, the introduction of object-oriented methods was necessary. RF-4 or now GeoSys was (again) completely re-designed and rewritten in C++
\cite{oK04}, \cite{wW06}.
Several doctoral theses have been prepared in the fields of geotechnical simulation (DECOVALEX project, \cite{Eng:03}, \cite{deJ:2004}, \cite{Wal:2007}), contaminant hydrology (Virtual Aquifer project, \cite{Bei:2005}, \cite{Bey:2007}, \cite{Mil:2007}), geothermal reservoir modeling (Urach Spa project, \cite{Ten:2006}).
Aside computational mechanics progress had been made as well in the pre-processing for numerical analysis \cite{Kal:2006}, \cite{Gro:2006}, \cite{Che:2006}.
First GeoSys/RockFlow habilitations appeared \cite{Bau:2006}, \cite{McD:2006}, \cite{Kos:2007}.
As mentioned in the beginning it is impossible to cite everything, other important works in the Tuebingen era are \cite{ParEtAl:2007}, \cite{KolEtAl:2007}, \cite{XieEtAl:2006}.

\subsubsection*{GeoSys/OpenTHMC}

Leipzig:
The new challenge for GeoSys is to continue the development as a distributed open-source project, i.e. sharing and widening the knowledge, as people from the GeoSys group got interesting offers. The number of  GeoSys-project partners is already quite large (see cover page).
At the Helmholtz Center for Environmental Research a new research platform TESSIN is available, which combines high-performance-computing (HPC) and visualization facilities. Post-processing becomes more and more important as more and more information becomes available, due to high-resolution measurement techniques and HPC itself \cite{WanEtAl:2008}.

%-------------------------------------------------------------------------
\subsection*{Next ...}

Benchmarks coming soon ...
\begin{itemize}
  \item Matrix diffusion (CMCD, GK)
  \item Thermal signatures in soils and sediments (JOD)
  \item Gas flow (OK)
  \item Heat transport in gas flows (OK)
  \item Two-phase flow and CO2 stuff (PCH)
  \item Biodegradation et al (Kiel: CB, SB)
  \item More overland flow (EK, SF)
  \item ...
\end{itemize}

