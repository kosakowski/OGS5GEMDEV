\subsection{Background}
\subsubsection*{Mass balance equation}
Consider two-phase flow in porous media, e.g liquid (denoted by $l$) and gas (denoted by $g$). For each phase in two-phase fluid flow, mass conservation is given by the following equation:

\begin{align}
\pD{}{t}\left(\poro S^g\dens_k^g+\poro S^l\dens_k^l\right)+\nabla \cdot \left( \FlxDf_k^g+\FlxDf_k^l\right)=Q_k
\label{eq:massb}
\end{align}

where $S$ is saturation, $\dens$ stands for phase density, $\poro$ is the porosity, $\FlxDf$ is total flux. The subscript $k$ in equation (\ref{eq:massb}) denotes the component, e.g air ($k=a$) or water ($k=w$), within each phase, $\gamma=(g,l)$. For any phase $\gamma=(g,l)$, an advection vector ${{\FlxDf}_A}_k^{\gamma}$ and a diffusion vector  ${{\FlxDf}_D}_k^{\gamma}$ comprise the total flux, i.e

\begin{align}
\FlxDf_k^{\gamma}={{\FlxDf}_A}_k^{\gamma}+{{\FlxDf}_D}_k^{\gamma}
\label{eq:tflx}
\end{align}

According to Darcy's law, the advective part of the total flux may be written as

\begin{align}
{{\FlxDf}_A}_k^{\gamma}=-\dens_k^{\gamma}\dfrac{\perm \RelKa^{\gamma}}{\mu^{\gamma}}\left(\nabla \pres^{\gamma}-\dens^{\gamma} \mathbf g\right)
\label{eq:flx_dc}
\end{align}

where $\perm$ is the intrinsic permeability, $\RelKa^{\gamma}$ is the relative permeability of the phase, and $\mu^{\gamma}$ is the viscosity.

The diffusive part of the total flux is given by Fick's law

\begin{align}
{{\FlxDf}_D}_k^{\gamma}=-\poro \sat^{\gamma}  \dens^{\gamma} {\mathbb D}_k^{\gamma} \nabla \left(\dfrac{\dens_k^{\gamma}}{\dens^{\gamma}}\right)
\label{eq:flx_fk}
\end{align}

where $\mathbb D$ is the diffusion coefficient tensor. Since $\dens^{\gamma} = \dens_a^{\gamma}+\dens_w^{\gamma}$, we have

 \begin{align}
{{\FlxDf}_D}_w^{\gamma}+{{\FlxDf}_D}_a^{\gamma}=\mathbf 0
\label{eq:dufblc}
\end{align}

under the assumption ${\mathbb D}_a^{\gamma}  = {\mathbb D}_w^{\gamma} $.

Consider a water-air mixture. We expand the mass balance equation (\ref{eq:massb}) with the flux defined in equations (\ref{eq:tflx}) based upon the above equations (\ref{eq:tflx}, \ref{eq:flx_dc}, \ref{eq:flx_fk}). For the water component, the diffusive part of the total flux takes the form

\begin{align}
{{\FlxDf}_D}_w^{l}=-\poro \Sat^{l}  \dens^{l} {\mathbb D}_w^{l} \nabla \left(\dfrac{\dens_w^{l}}{\dens^{l}}\right),\quad
{{\FlxDf}_D}_w^{g}=-\poro \Sat^{g}  \dens^{g} {\mathbb D}_w^{g} \nabla \left(\dfrac{\dens_w^{g}}{\dens^{g}}\right)
\label{eq:flx_fkw}
\end{align}

Obviously, ${\mathbb D}_w^{l} = \mathbf 0$. Therefore, the mass balance equation for water component can be written as follows

\begin{align}
\pD{}{t} \left(\poro S^g\dens_w^g+\poro S^l\dens_w^l\right)-
\nabla \cdot \left[\dens_w^{l}\dfrac{\perm \RelKa^{l}}{\mu^{l}}\left(\nabla \pres^{l}-\dens^{l} \mathbf g\right)\right]\nonumber\\
-\nabla \cdot \left[\dens_w^{g}\dfrac{\perm \RelKa^{g}}{\mu^{g}}\left(\nabla \pres^{g}-\dens^{g} \mathbf g\right)\right] -
\nabla \cdot \left[\poro \Sat^{g}  \dens^{g} {\mathbb D}_w^{g} \nabla \left(\dfrac{\dens_w^{g}}{\dens^{g}}\right)\right] = Q_w
\label{eq:massblq}
\end{align}

Since the capillary pressure $\pres^c$  is chosen as one of the two unknowns of equation (\ref{eq:massb}) and $S^g=1-S^l$, equation (\ref{eq:massblq}) becomes

\begin{align}
\poro (\dens_w^l -\dens_w^g)\pD{S^l}{t} +(1 -S^l)\poro \pD{\dens_w^{g}}{t} -
\nabla \cdot \left[\dens_w^{l}\dfrac{\perm \RelKa^{l}}{\mu^{l}}\left(\nabla (\pres^{g}-\pres^{c}) -\dens^{l} \mathbf g\right)\right]\nonumber\\
-\nabla \cdot \left[\dens_w^{g}\dfrac{\perm \RelKa^{g}}{\mu^{g}}\left(\nabla \pres^{g}-\dens^{g} \mathbf g\right)\right] -
\nabla \cdot \left[\poro \Sat^{g}  \dens^{g} {\mathbb D}_w^{g} \nabla \left(\dfrac{\dens_w^{g}}{\dens^{g}}\right)\right] = Q_w
\label{eq:msblq}
\end{align}

Similar to the previous procedure, the diffusion part of the total flux of air component can be written as

\begin{align}
{{\FlxDf}_D}_a^{l}=-\poro \Sat^{l}  \dens^{l} {\mathbb D}_a^{l} \nabla \left(\dfrac{\dens_a^{l}}{\dens^{l}}\right),\quad
{{\FlxDf}_D}_a^{a}=-\poro \Sat^{g}  \dens^{g} {\mathbb D}_a^{g} \nabla \left(\dfrac{\dens_a^{g}}{\dens^{g}}\right)
\label{eq:flx_fka}
\end{align}

The density shift from air component to liquid ${\dens_a^{l}}$ is very small and can be omitted. Therefore, we can assume ${{\FlxDf}_D}_a^{l}\thickapprox0$. As a consequence, the mass balance equation for air component is derived as:

$$\pD{}{t} \left(\poro S^g\dens_a^g\right) -$$
\begin{align}
\nabla \cdot \left[\dens_a^{g}\dfrac{\perm \RelKa^{g}}{\mu^{g}}\left(\nabla \pres^{g}-\dens^{g} \mathbf g\right)\right]-\nabla \cdot \left[\poro \Sat^{g}  \dens^{g} {\mathbb D}_a^{g} \nabla \left(\dfrac{\dens_a^{g}}{\dens^{g}}\right)\right] =Q_a
\label{eq:massba}
\end{align}

Expanding the temporary derivative term of equation (\ref{eq:massba}) yields

$$-\poro \dens_a^g \pD{S^l}{t} + (1 -S^l)\poro \pD{\dens_a^{g}}{t}-$$
\begin{align}
\nabla \cdot \left[\dens_a^{g}\dfrac{\perm \RelKa^{g}}{\mu^{g}}\left(\nabla \pres^{g}-\dens^{g} \mathbf g\right)\right] -
\nabla \cdot \left[\poro \Sat^{g} \dens^{g} {\mathbb D}_a^{g} \nabla \left(\dfrac{\dens_a^{g}}{\dens^{g}}\right)\right] = Q_a
\label{eq:msba}
\end{align}

The mass balance equations (\ref{eq:msblq}) and (\ref{eq:msba}) are exactly the same as described in \cite{SanPesSch:06}.

\subsubsection*{Pressure-pressure (pp) scheme}
Based on the description of isothermal two-phase flow above, (\ref{eq:msblq}) and (\ref{eq:msba}) can be modified in order to obtain governing equation for the isothermal two-phase flow in a porous medium. In this formulation primary variables are gas pressure $\asup{\pres}{g}$, and capilary pressure $\asup{\pres}{c}$.

The basic equations of the isothermal two-phase flow system are:

\begin{align}
\poro \dens_w \pD{S_w}{\pres_c} \dot\pres_c +
\nabla \cdot\left[\dens_w\dfrac{{\perm \RelKa}_{w}}{\mu_w}\left(-\nabla \pres^{g} +
\nabla{\pres}^{c} + \dens_w \mathbf g\right)\right] = Q_w
\end{align}
\begin{align}
- \poro \dens^a \pD{S_w}{\pres_c} \dot\pres_c+
\poro (1 -S_w)\left(\pD{\dens_a}{\pres^g}\dot\pres^g+\pD{\dens_a}{\pres_c}\dot\pres_c\right)+ \nonumber\\
\nabla \cdot \left[\dens_a\dfrac{{\perm \RelKa}_a}{\mu_a} \left(-\nabla\pres^{g} + \dens_a \mathbf g\right) \right] = Q_a
\label{eq:msbl_sim}
\end{align}

\subsubsection*{Pressure-saturation (pS) scheme}
Based on the description of the isothermal two-phase flow above, (\ref{eq:msblq}) and (\ref{eq:msba}) can be modified in order to obtain governing equation for the isothermal two-phase system. Primary variables of this formulation are wetting phase pressure $p_w$, and non-wetting phase saturation $S_{nw}$. The equations are simply algebraic manipulations of those in the previous section.