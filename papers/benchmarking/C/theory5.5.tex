
\textbf{Theory}

Diffusion is a process that equates concentration differences of gaseous or dissolved matter or energy. The particles move from higher to lower concentrations by Brownian movement in dependence on the temperature. In an aquifer, diffusive transport appears when convective transport is not that relevant (small velocities).

The extent of diffusion is also dependent on the diffusing substance and the medium. In addition, diffusion in soils is influenced by other factors, e.g. tortuosity. The finer a soil the stronger are the interacting forces between the soil matrix and the diffusing molecules. The diffusion coefficient which has to be given in GeoSys/RockFlow is the so-called apparent diffusion coefficient (eq. \ref{eq511}).
\begin{equation}
D_a\,=\,\frac{D_e}{\Phi}
\label{eq511}
\end{equation}
{\small
with $D_e$ - effective diffusion coefficient.
}
