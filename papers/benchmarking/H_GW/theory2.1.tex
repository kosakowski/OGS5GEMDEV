
\textbf{Theory}

Water flow in a saturated porous medium is influenced by the pressure gradient over a given distance and the hydraulic conductivity of the aquifer. By Darcy's Law (equ. \ref{eq21}) the flow rate by considering these influences can be calculated.
\begin{equation}
v_f\,=\,k_f\cdot i
\label{eq21}
\end{equation}
{\small
with
\begin{itemize}
\item[$v_f$] -- flow rate (m/s),
\item[$k_f$] -- hydraulic conductivity (m/s),
\item[$i$] -- pressure gradient (-).
\end{itemize}
}

The hydraulic conductivity is calculated by the following relation.
\begin{equation}
k_f\,=\,\frac{\kappa\cdot\rho\cdot g}{\mu}
\label{eq22}
\end{equation}
{\small
with
\begin{itemize}
\item[$\kappa$] -- permeability (m$^2$)
\item[$\rho$] -- density of the fluid (kg$\cdot$m$^{-3}$)
\item[$g$] -- gravity constant (m$\cdot$s$^{-2}$)
\item[$\mu$] -- dynamic viscosity of the fluid (Pa$\cdot$s)
\end{itemize}
}

By using the law of continuity the discharge through a defined cross section can be calculated.
\begin{equation}
Q\,=\,v_f\cdot A
\label{eq23}
\end{equation}
{\small
with
\begin{itemize}
\item[$Q$] -- discharge (m$^3$/s)
\item[$A$] -- cross section (m$^2$)
\end{itemize}
}

Layered soil material is possibly less permeable in one direction than in the direction perpendicular to it. In this case the input of different values for the permeability $\kappa$ in dependence on the direction of anisotropy is possible in RockFlow.
