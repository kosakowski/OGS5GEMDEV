\chapter*{Preface}
\thispagestyle{empty}
%\addcontentsline{toc}{chapter}{Preface}

Fluids play an important role in environmental systems appearing
as surface water in rivers, lakes, and coastal regions or in the
subsurface as well as in the atmosphere. Mechanics of
environmental fluids is concerned with fluid motion as well as
associated mass and heat transport. In subsurface systems we have
additionally to deal with deformation processes in soil and rock
systems.

%Contents
This textbook is organized in four sections. The first part gives
the background of continuum mechanics. We consider the general
balance equations of mass, momentum and energy, averaging concepts
for turbulence and the porous medium approach. The second part is
dealing with numerical methods for solving partial differential
equations. The basic concepts of approximation theory and
afterwards finite differences, finite elements as well as finite
volumes are explained and illustrated with basic equations for
diffusion, advection and transport processes. In the third part of
this book aspects of implementation numerical methods in an
object-oriented framework are discussed. Object-oriented
programming (OOP) has become exceedingly popular in the past few
years. OOP is more than rewriting programs in modern languages,
OOP is a new way of thinking about designing and realizing
software projects.

The first three parts of this book were used for a master course
in computational fluid mechanics. In the fourth section several
topics about recent research in fractured-porous media modeling
are presented (e.g. non-linear flow, heat transport,
density-dependent flow, multiphase flow and deformation
processes). These topics, in particular on non-linear problems,
might be useful for PhD courses in computational mechanics.

%History
The material of this textbook I prepared during my work at the
Institute of Fluid Mechanics at the University of Hannover from
1994-2001. My lectures on computational fluid mechanics
(Str�mungsmechanik V/VI) were part of a more general lecture
circle on fluid mechanics by Prof. W. Zielke, J. Strybny, R. Ratke
(Str�mungsmechanik I/II), Prof. M. Markofsky (Str�mungsmechanik
III/IV) for civil engineering students (Roman numbers indicate
semesters). The idea to prepare a textbook in English was to offer
this material also to Master and PhD courses in civil and
environmental engineering. The lectures were part of the teaching
program in civil engineering at the International Centre of
Computational Engineering Science (ICCES) in Hannover
(\texttt{www.icces.de}) and now in environmental engineering at
the Centre of Applied Geosciences in T�bingen (ZAG)
(\texttt{www.uni-tuebingen.de/geo/zag/}).

%Acknowledgement
I am indebted to many of my colleagues and students at the
Institute of Fluid Mechanics in Hannover. In particular, I thank
Prof. Werner Zielke for continuously supporting my work and my PhD
students Carsten Thorenz, Ren� Kaiser, Abderrahmane Habbar, Thomas
Rother, Martin Kohlmeier and Sylvia Moenickes (in the order of
appearance) for their enthusiastic work concerning the RockFlow
project (\texttt{www.rockflow.de}). The assistance by Martin
Beinhorn and J\"oelle de Jonge to improve the English is also
appreciated. A special thank is extended to Prof. Hans-J�rg
Diersch (Berlin, \texttt{www.wasy.de}) for his useful comments on
this book. I appreciate the valuable discussions with Profs. T.
Taniguchi (Okayama), B. Berkowitz (Rehovot), G. Starke (Hannover),
T. Schanz (Weimar), M. Sauter (Jena) and G. Teutsch (T�bingen) on
theoretical and practical aspects of fluids in the environment. I
would also like to thank the Springer-Verlag for realizing this
publishing project and, especially, Dr. Ditzinger for encouraging
me to spend the time this summer to transform the lecture script
into a textbook.

The heaviest burden involved in writing this book was borne by my
wife, Barbara, who had to put up with the many inconveniences that
are unavoidable when one is engaged in writing a book. I am very
grateful to her for the continuous encouragement. Finally, I would
like to dedicate this book to my family without whom it might be
never have been completed.


\bigskip
Hannover / T�bingen, August 2001 \hfill Olaf Kolditz
