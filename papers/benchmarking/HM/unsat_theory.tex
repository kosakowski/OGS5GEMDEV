\subsection{Theory}

\subsubsection{Fluid mass balance}

The general fluid mass balance equation for unsaturated flow in a deformable porous medium is
%
\begin{eqnarray}
n S^l \dot\rho^l
+
n \rho^l \dot S^l
+
n S^l \rho^l \v^{ls}
+
S^l \rho^l \nabla\cdot\v^s
+
S^l \rho^l \frac{1-n}{\rho^s} \dot\rho^s
=
\nonumber \\
Q_\rho^l + S^l \frac{\rho^l}{\rho^s} Q_\rho^s
\label{eqn:uc1}
\end{eqnarray}
%
Hereby, the basic assumption of Richards type models is that the gaseous phase is immobile, i.e. $\v^g=0$.
Assuming grain incompressibility for isothermal conditions (i.e. $\dot\rho^s=0$) and no solid sources (e.g. resulting from chemical reactions) as well as
applying the constitutive equations for fluid compressibility, capillary pressure, and Darcy flux,
we obtain the following Richards equation for an unsaturated deformable porous medium.
%
\begin{eqnarray}
n S^l \rho_0^l \beta_p \dot p^l
-
n \rho^l \frac{dS^l}{d p_c} \dot p^l
-
\nabla
\left(
\dfrac{\RelKa^l\per}{\mu^l}(\nabla p^l - \dens\grv)
\right)
+
S^l \rho^l \nabla\cdot \dot\u^s
=
Q_\rho^l
\label{eqn:uc2}
\end{eqnarray}
%
Rearrangement with respect to the primary variables $p^l, \u^s$ yields
%
\begin{eqnarray}
\left(
-
n \rho^l \frac{dS^l}{d p_c}
+
n S^l \rho_0^l \beta_p
\right)
\dot p^l
-
\nabla
\left(
\dfrac{\RelKa^l\per}{\mu^l} \nabla p^l
\right)
+
S^l \rho^l \nabla\cdot \dot\u^s
=
\nonumber\\
Q_\rho^l
+
\nabla
\left(
\dfrac{\RelKa^l\per}{\mu^l}\dens\grv
\right)
\label{eqn:uc2}
\end{eqnarray}
%

A constitutive equation, the water content function obtained by experiments, characterizes the relationship between $p^l$ and $\sat^l$ and therefore the derivative $dS^l/dp_c$.

\subsubsection{Momentum balance}

The deformation process is described by the momentum balance equation for the unsaturated porous medium in terms of stresses.

\begin{eqnarray}
\nabla\cdot
(\s - \alpha_b S^l p^l \bf I) + \rho\g = 0
\label{eqn:mb_us}
\end{eqnarray}

All symbols are denoted in chapter \ref{sec:symbols}.
%

\subsubsection{Numerical scheme}

The standard Galerkin finite element approach is applied for the numerical solution of the PDEs (\ref{eqn:uc2}) and (\ref{eqn:mb_us}) resulting into the following system of algebraic equations.

\begin{eqnarray}
\underbrace{
\left[
\begin{array}{ll}
\mathbf{C}_{pp} & \mathbf{C}_{pu}
\\
\mathbf{C}_{up} & \mathbf{C}_{uu}
\end{array}
\right]
}_{\mathbf C}
%
\frac{d}{dt}
\underbrace{
\left\{
\begin{array}{l}
\hat\PressureVec^l
\\
\hat\Disp^s
\end{array}
\right\}
}_{\mathbf x}
%
+
%
\underbrace{
\left[
\begin{array}{ll}
\mathbf{K}_{pp} & \mathbf{K}_{pu}
\\
\mathbf{K}_{up} & \mathbf{K}_{uu}
\end{array}
\right]
}_{\mathbf K}
%
\left\{
\begin{array}{l}
\hat\PressureVec^l
\\
\hat\Disp^s
\end{array}
\right\}
%
=
\underbrace{
\left\{
\begin{array}{l}
\mathbf{r}_p
\\
\mathbf{r}_u
\end{array}
\right\}
}_{\mathbf{r}}
\nonumber
\\
\end{eqnarray}

where $\C_{up}, \C_{uu}, \K_{pu}$ are zero.

\begin{eqnarray}
\left[
\begin{array}{cc}
\mathbf{C}_{pp} & \mathbf{C}_{pu}
\\
\mathbf{0} & \mathbf{0}
\end{array}
\right]
%
\frac{d}{dt}
\left\{
\begin{array}{l}
\hat\PressureVec^l
\\
\hat\Disp^s
\end{array}
\right\}
%
+
%
\left[
\begin{array}{ll}
\mathbf{K}_{pp} & \mathbf{0}
\\
\mathbf{K}_{up} & \mathbf{K}_{uu}
\end{array}
\right]
%
\left\{
\begin{array}{l}
\hat\PressureVec^l
\\
\hat\Disp^s
\end{array}
\right\}
%
=
\left\{
\begin{array}{l}
\mathbf{r}_p
\\
\mathbf{r}_u
\end{array}
\right\}
\nonumber
\\
\end{eqnarray}

%------------------
Time discretization with explicit finite differences yields

\begin{eqnarray}
\left(
\frac{1}{\Delta t} \C_{pp} + \theta \K_{pp}
\right)
\hat{\p}^l_{n+1}
+
\frac{1}{\Delta t} \C_{pu}
\hat\u^s_{n+1}
\nonumber\\
=
\left(
\frac{1}{\Delta t} \C_{pp} + (1-\theta) \K_{pp}
\right)
\hat\p^l_{n}
+
\frac{1}{\Delta t} \C_{pu}
\hat\u^s_{n}
+
\r_p
\end{eqnarray}

%------------------

\begin{eqnarray}
\theta \K_{up}
\hat\p^l_{n+1}
+
\theta \K_{uu}
\hat\u^s_{n+1}
\nonumber\\
=
(1-\theta) \K_{up}
\hat\p^l_{n+1}
+
(1-\theta) \K_{pp}
\hat\u^s_{n+1}
+
\r_u
\end{eqnarray}

%------------------

with following finite element matrices

\begin{eqnarray}
%---
\C^e_{pp}
&=&
\int_{\Domain^e}
\SFVPressure^T
\left[
- n \rho^l \frac{dS^l}{dp_c}
+
n S^l \rho_0^l \beta_p
%\UnitOperator^T\GradOperator
\right]
\SFVPressure
\,
d\Domain^e
\nonumber
\\
%---
\K^e_{pp}
&=&
\int_{\Domain^e}
\nabla\SFVPressure^T
\left[
- \dfrac{\RelKa^l\per}{\mu^l}
\right]
\nabla\SFVPressure
\,
d\Domain^e
\nonumber
\\
%---
\C^e_{pu}
&=&
\int_{\Domain^e}
\SFVPressure^T
\left[
S^l \rho^l
\right]
\nabla\TFVDisp
\,
d\Domain^e
%\nonumber
\\
%---
\r^e_{p}
&=&
\int_{\Domain^e}
\left[
Q^l_\rho
\right]
\SFVPressure
\,
d\Domain^e
+
\int_{\Domain^e}
\SFVPressure^T
\left[
\dfrac{\RelKa^l\per}{\mu^l} \rho^l \g
\right]
\nabla\SFVPressure
\,
d\Domain^e
\nonumber
\end{eqnarray}

%------------------

\begin{eqnarray}
%---
\K^e_{up}
&=&
\int_{\Domain^e}
\B^T
\left[
\alpha_b S^l
\right]
\m\SFVPressure
\,
d\Domain^e
\nonumber
\\
%---
\K^e_{uu}
&=&
\int_{\Domain^e}
\B^T
\left[
\mathbb C
\right]
\B
\,
d\Domain^e
\nonumber
\\
%---
\r^e_{u}
&=&
\int_{\Domain^e}
\TFVDisp
\left[
\rho\g
\right]
\,
d\Domain^e
\end{eqnarray}

where m is a mapping vector as
\[\m = [1\, 1\, 1\, 0]^T\] 
for plane strain problems and
\[\m = [1\, 1\, 1\, 0\, 0\, 0]^T\] for 3D problems
%%%%%%%%%%%%%%%%%%%%%%%%%%%%%%%%%%%%%%%%%%%%%%%%%%%%%%%%%%%%%%%%%%%%%%%%%%
