\section{Theory}
We consider flow of an incompressible fluid (liquid) in a
deforming porous medium. The system consists of two phases (liquid
and solid phase). The unknown field functions are liquid pressure
$\Pressure$ and solid displacement $\Disp$. For the
determination of the unknown field functions we combine the mass
balance equations and the momentum balance equation of fluid and
solid phases (\cite{LewRobSch:89},  \cite{Kol:02}).

The deformation process is modelled by the system
\begin{align}
  \nabla \cdot (\Str - \Pressure \bI) + \rho \bg
  &=\bzero\mbox{ in }\Omega \: ,
  \label{eq:lsbiot-system-elas1} \\
  \Str - \ElasticityTensor \bepsilon(\Disp) &=\bzero\mbox{ in }\Omega \: ,
  \label{eq:lsbiot-system-elas2}
\end{align}
where $\Str$ denotes the effective stress,
and the momentum balance equation is modified such that it
incorporates stresses and forces connected to the fluid pressure.
%Note that the displacement field $\Disp$ of this model is defined with respect
%to the equilibrium under gravity. The true effective stress in the porous
%medium may be computed from $\Str$ by adding the effect of its own
%weight due to gravity.
Here, $\bI$ denotes the identity, $\GravityVector$ denotes the
gravity vector and $\ElasticityTensor$ is the elasticity tensor
representing the dependence of strain and stress for a linear
elastic material law.


Fluid flow in deformable porous media is described by the following mass and
momentum balance equations,
\begin{align}
  \nabla \cdot \FluxFluid + \nabla \cdot \frac{\partial \Disp}{\partial t}
  & = 0 \mbox{ in } \Omega \: ,
  \label{eq:lsbiot-system-flow1} \\
  \FluxFluid + \frac{\bk}{\Viscosity} \: (\nabla \Pressure - \Density \bg)
  &=\bzero\mbox{ in }\Omega \: .
  \label{eq:lsbiot-system-flow2}
\end{align}
In consolidation theory, the fluid mass balance equation
(\ref{eq:lsbiot-system-flow1}) has an extra term due to the
deformation process. The second equation
(\ref{eq:lsbiot-system-flow2}) is denoted as the Darcy law. Here,
$\FluxFluid$ denotes the volumetric flux of the fluid and
$\Pressure$ the liquid phase pressure. $\PermTensor$ denotes the
permeability tensor, $\Viscosity$ is the liquid viscosity, and
$\Density$ is the density of the porous medium. Hereby we assume
solid grains itself are incompressible, i.e. $d^s\rho/d^s t = 0$

After applying Galerkin finite element approach to eqns . (\ref{eq:lsbiot-system-elas1}) and (\ref{eq:lsbiot-system-flow1}) build a set of
coupled linear equations to be solved for the primary variables
liquid pressure $\Pressure$ and solid deformation $\Disp$ of the
$\Pressure / \Disp$ formulation of the consolidation problem. The
resulting equation system can be compactly written in following
form
\begin{equation}
\left[%
\begin{array}{cc}
\bK_{pp}^* & \frac{1}{\Delta t}\bC_{u}^*
\\
\bK_{up}^* & \bK_{uu}^*
\end{array}%
\right]
\,
\left[%
\begin{array}{c}
\hat{\bp}^{i+1}
\\
\hat{\bu}^{i+1}
\end{array}%
\right]
=
\left[%
\begin{array}{l}
\mathbf{f}_{p}^* + \frac{1}{\Delta t}\bC_{u}^* \, \hat\bu^i
\\
\mathbf{f}_{u}^*
\end{array}%
\right]
\label{eqn:pu_eqn}
\end{equation}

The above algebraic equation system (\ref{eqn:pu_eqn}) for
determination of the primary variables of the $p/\Disp$
formulation, i.e. nodal pressures and displacements.
