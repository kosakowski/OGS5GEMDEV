For small strains of solid continua, it is mostly justified to assume isotropic elastic material behavior. Most substantial details of the theory of linear isotropic elasticity, represented by the generalized Hooke's law, are discussed in Sec.~\ref{sec:elasticity}.

In many technical applications considering small strains, the elastic material parameters are assumed to be constant, and the stress-strain curves are nearly linear. However, the typical response of certain geological materials to monotonic loading (without load reversal) shows a nonlinear stress-strain behavior. Considering only elastic effects during load application, Hooke's law cannot be used to describe the observed material properties. Therefore, so-called pseudo-elastic constitutive models are frequently used for the analysis of nonlinear stress-strain curves, particularly in soil and rock mechanics. In a generalized manner, they are based on the assumption of an explicit stress-strain relation considering a stress- and strain-dependent material matrix:
\begin{equation}
\mio{\sigma}{}{}\,=\,\fourtens{C}(\mio{\sigma}{}{},\mio{\varepsilon}{}{})\ccdot\miu{\varepsilon}{\mathrm{el}}{}\,.
\label{Me_elasticity_nonlin}
\end{equation}

Based on the so-called {\sl Lubby1} model (cf. \cite{Lux:1984}), a nonlinear elastic approach with strain-dependent Young's modulus
\begin{equation}
E(\varepsilon_{\mathrm{v}})\,=\,\frac{E_0}{1+a\,\varepsilon_{\mathrm{v}}^n}
\label{Me_lubby1_ev}
\end{equation}
but constant Poisson's ratio is proposed. Here, $\varepsilon_{\mathrm{v}}$ is the equivalent strain, and $E_0$, $a$ as well as $n$ are material parameters . The equivalent strain is defined by
\begin{equation}
\varepsilon_{\mathrm{v}}\,=\,
\sqrt{\frac{2}{3}\,\miu{\varepsilon}{\mathrm{el}}{}\ccdot\miu{\varepsilon}{\mathrm{el}}{}}\,.
\end{equation}


If the material properties are independent of orientations and directions of the technical or natural object under consideration, the material behavior is called {\sl isotropic}. Otherwise, the material is known as {\sl anisotropic} one. Anisotropy is closely connected with distinguished orientations in the material structure. Among others, fiber-reinforced and layered materials are typical anisotropic materials.

From the the point of view of modeling and numerical simulation special cases of anisotropy like {\sl orthotropy} are of particular interest. Orthotropic materials are characterized by mutually orthogonal two-fold axes of rotational symmetry. A special class of orthotropic materials represent the so called {\sl transverse isotropic} materials. They are characterized by a plane of isotropy featuring the same material properties independent of the direction of observation within this plane, and different material properties in the direction normal to this plane. Within this context, the normal to the plane of isotropy can be considered as the direction of anisotropy. Most of layered materials, biological membranes as well as rocks (e.\,g. sandstone, shale) are typical materials which can be considered as transverse isotropic ones.

In case of transverse isotropy, the Hooke's law (\ref{eq:hooke_isotherm}) has to be modified establishing a unit vector $\miu{a}{}{}$ which defines the direction perpendicular to the plane of isotropy (normal vector, direction of anisotropy -- defining, e.\,g., the direction of a single fiber family of a fiber-reinforced material).
\begin{eqnarray}
\sigma_{ij}& = & \lambda\,\delta_{ij}\,\varepsilon_{kk}\,+\,2\mu_T\,\varepsilon_{ij} \nonumber \\[1.50ex]
 &  & +\,2\,\left(\mu_L-\mu_T\right)\,\left(a_i\,\varepsilon_{jl}\,a_l+a_l\,\varepsilon_{li}\,a_j\right) 
 \nonumber \\[1.50ex]
 &  & +\,\alpha\,\left(a_i\,a_j\,\varepsilon_{kk}+a_k\,\varepsilon_{kl}\,a_l\,\delta_{ij}\right) 
 \nonumber \\[1.50ex]
 &  & +\,\beta\,a_k\,\varepsilon_{kl}\,a_l\,a_i\,a_j
\label{Me_hooke_transviso}
\end{eqnarray}
Linear elastic transverse isotropic material is characterized by 5 independent material parameters like $\lambda$, $\mu_T$, $\mu_L$, $\alpha$ and $\beta$ given in Eqn. (\ref{Me_hooke_transviso}). In some cases these parameters are called {\sl invariants} of the transverse isotropic elastic Hooke's law. They can be defined w.l.o.g. by the following (engineering) elastic constants which can be obtained experimentally: 

\begin{center}
\begin{tabular}{p{0.08\textwidth}p{0.025\textwidth}p{0.8\textwidth}}
$E_i$      & -- & Young's modulus within the plane of isotropy, \\[0.5ex]
$\nu_i$    & -- & Poisson's ratio within the plane of isotropy, \\[1.5ex]
$E_a$      & -- & Young's modulus w.r.t. the direction of anisotropy, \\[0.5ex]
$\nu_{ia},\,\nu_{ai}$ & -- & Poisson's ratio w.r.t. the direction of anisotropy, \\[0.5ex]
$G_a$      & -- & shear modulus w.r.t. the direction of anisotropy.
\end{tabular}
\end{center}

There exist some relations between these parameters.
\begin{eqnarray}
G_i      & \!\!\!\!\!= & 
\!\!\!\!\!\ttfrac{E_i}{2(1+\nu_i)}=\mu_i\;\;\mbox{(shear modulus within the plane of isotropy)} \\[2.0ex]
\nu_{ai} & \!\!\!\!\!= & \!\!\!\!\!\nu_{ia}\,\ttfrac{E_a}{E_i}
\end{eqnarray}

As mentioned above, the invariants of the transverse isotropic elastic Hooke's law can be expressed by the presented elastic parameters.
\begin{eqnarray*}
\lambda & = &  \ttfrac{E_i(\nu_i+\nu_{ia}\nu_{ai})}{{\widetilde D}} \\[2.0ex]
\mu_T   & = &  G_i \\[2.0ex]
\mu_L   & = &  G_a \\[2.0ex]
\alpha  & = &  \ttfrac{E_i(\nu_{ai}(1+\nu_i-\nu_{ia})-\nu_i)}{{\widetilde D}} \\[2.0ex]
\beta   & = &  \ttfrac{E_a(1-\nu_i^2)-E_i[(\nu_i+\nu_{ia}\nu_{ai})+2(\nu_{ai}(1+\nu_i-\nu_{ia})-\nu_i)]}
                      {{\widetilde D}} \\
        &   & \,-\,4G_a\,+\,2G_i \\[4.0ex]
        &   &  \mbox{with}\quad {\widetilde D}\,=\,
               1\,-\,\nu_i^2\,-\,2\,\nu_{ia}\,\nu_{ai}\,-\,2\,\nu_{ia}\,\nu_{i}\,\nu_{ai} \nonumber \\[1.0ex]
        &   &  \mbox{\hspace*{9.0ex}} =\,
               (1\,+\,\nu_i)(1\,-\,\nu_i\,-\,2\,\nu_{ia}\,\nu_{ai})
\end{eqnarray*}

The coordinates of the material tensor for linear elastic transverse isotropic material are defined as follows:
\begin{eqnarray}
{C}_{ijkl} & = & \lambda\,\delta_{ij}\,\delta_{kl}\,+\,
                        2\mu_T\,\delta_{ik}\,\delta_{jl} \nonumber \\[1.50ex]
                   &   & +\,2\,\left(\mu_L-\mu_T\right)\,
                         \left(a_i\,\delta_{jk}\,a_l+a_k\,\delta_{il}\,a_j\right) \nonumber \\[1.50ex]
                   &   & +\,\alpha\,\left(a_i\,a_j\,\delta_{kl}+a_k\,a_l\,\delta_{ij}\right) \nonumber \\[1.50ex]
                   &   & +\,\beta\,a_i\,a_j\,a_k\,a_l
\end{eqnarray}

\clearpage