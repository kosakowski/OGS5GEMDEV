This chapter is dedicated to the analysis of pure deformation processes in solid continua. Within the context of porous media mechanics, the generalized local momentum balance Eq.~(\ref{eq:4}) dicussed in Sec.~\ref{sec:momentum_balance} serves as governing equation describing mechanical deformation. In fact, the specific expression of the momentum balance Eq.~(\ref{eq:4}) defines the equilibrium conditions in porous media (here, considering swelling and thermal stresses caused by the coupling of mechanical to other physical and chemical processes). The effective stress principle has been established in order to define the stress state in the solid skeleton of porous media (cf. Sec.~\ref{sec:effstress}). Within this context $\miu{\sigma}{\mathrm{eff}}{}$ indicates the stress tensor applied to a substitute continuum representing the solid skeleton smeared over the volume of the porous medium under consideration, and being characterized by a reduced partial density compared to the material density of the solid skeleton. Material models, which are well-known from solid mechanics are transfered directly to the description of the material behavior of the solid skeleton in porous media mechanics.

Assuming small strains, the equilibrium conditions in solid mechanics are defined by the following specific formulation of the balance of linear momentum:
\begin{equation}
\nabla \cdot \miu{\sigma}{}{} + \rho\miu{g}{}{} = 0
\label{M_eq:momb}
\end{equation}
where $\miu{\sigma}{}{}$ is the Cauchy's stress tensor, $\rho$ is the mass density, and $\rho\miu{g}{}{}$ is the volume force with the gravity vector $\mio{g}{}{}$. The coefficients of the displacement vector $\miu{u}{}{}$ are the primary variables, which will become evident introducing an appropriate constitutive relation describing the specific stress-strain behavior of the material under consideration into the weak formulation of Eq.~(\ref{M_eq:momb}). For more details about the systematics af typical material classes see Sec.~\ref{sec:matclass}.

In general, the deformation problem can be considered as an initial-boundary value problem with Neumann type and Dirichlet type boundary conditions accordingly given by
\begin{equation}
\miu{\sigma}{}{}\ccdot\miu{n}{}{}= \miu{t}{}{}
\quad \mbox{or} \quad
\miu{u}{}{} = \miu{u}{\Gamma}{},
\quad \forall\,\Point \in
\partial \Omega
\label{M_eq:debc}
\end{equation}
where $\miu{n}{}{}$ defines the normal vector to the part of the surface with given traction boundary conditions $\miu{t}{}{}$, and $\miu{u}{\Gamma}{}$ are prescribed boundary displacement values.

Subsequently, the following benchmarks for deformation problems with increasing complexity (e.\,g., regarding the material behavior) are presented:

\medskip
{\sl Elasticity:}

\smallskip
\begin{compactitem}
	\item Plane strain confined compression (\ref{subsec:Me1})
	\item Plane strain confined compression -- Excavation in homogeneous media (\ref{subsec:Me2})
	\item Plane strain confined compression -- Excavation in heterogeneous media (\ref{subsec:Me3})
	\item Strain driven threedimensional unconfined compression (\ref{subsec:Me4})
	\item Load driven threedimensional unconfined compression (\ref{subsec:Me5})
	\item Nonlinear elastic axisymmetric triaxial compression (\ref{subsec:Me6})
	\item Transverse isotropic elastic tensile test (\ref{subsec:Me7})
\end{compactitem}

\medskip
{\sl Elastoplasticity:}

\smallskip
\begin{compactitem}
	\item Compression of a plate with a hole (\ref{subsec:Mp1})
	\item Twodimensional strain localization problem (\ref{subsec:Mp2})
	\item Cam-Clay plasticity (\ref{subsec:Mp3})
\end{compactitem}

\medskip
{\sl Viscoplastic creep:}

\smallskip
\begin{compactitem}
	\item Creep of a thick-walled cylinder (\ref{subsec:Mc1})
	\item Thermally driven creep in rock salt (\ref{subsec:Mc2})
	\item Stationary creep in rock salt (\ref{subsec:Mc3})
	\item Transient creep in rock salt (\ref{subsec:Mc4})
\end{compactitem}
