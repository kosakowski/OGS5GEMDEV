\subsection{Definition}

This problem shows 1D heat transport by advection and diffusion in a $\unit[100]{m}$ long fracture. The fracture is fully saturated with water, flowing with constant velocity. There is no rock matrix around the fracture considered which could store heat (this will be examined in the next example). Fig.~\ref{fig-addiff1} depicts the model set-up.
%
\begin{figure}[h]
\centering
\includegraphics[width=0.5\textwidth]{PART_II/T/Ad-Diff-Problem-def.eps}
\caption{\label{fig-addiff1}A fully saturated fracture with flowing water and a constant temperature at the left border}
\end{figure}

The fracture is described as a porous medium with $\unit[100]{\%}$ porosity, so that no solid material influences the heat transport process. The properties of the fluid are in Tab.~\ref{tab-addiff}.

\begin{table}[htbp!]
\caption{\label{tab-addiff}Model parameters}
\begin{center}
\begin{tabular}{llrr}
\toprule
Symbol & Parameter & Value & Unit \\
\midrule
$\rho^l$ & Density of water & $\unit[1000]{}$ & ${kg \cdot m^{-3}}$ \\			
$c^l$	& Heat capacity of water & $\unit[4000]{}$ & ${J \cdot kg^{-1} \cdot K^{-1}}$ \\
$\lambda^l$ & Thermal conductivity of water	& $\unit[0.6]{}$ & ${W \cdot m^{-1} \cdot K^{-1}}$ \\
$v$ & Water velocity & 3$\times 10^{-7}$ & $m/s$ \\
$L$ & Fracture length & 100 & $m$ \\
\bottomrule
\end{tabular}
\end{center}
\end{table}

These values cause a diffusivity constant for water of $\alpha=\unit[1.5 \cdot 10^{-7}]{m^2/s}$. The groundwater velocity in the fracture is $v=\unit[3.0 \cdot 10^{-7}]{m/s}$.

\subsection{Solution}

For 1D-advective/diffusive transport, an analytical solution is given by {Ogata $\&$ Banks} \cite{aO61} as
\begin{equation}
T(x,t)=\frac{T_0}{2}\bigg( \operatorname{erfc} \frac{x-v_x\cdot t}{\sqrt{4\alpha t}} + e^{\frac{v_x\cdot x}{\alpha}} \operatorname{erfc}\frac{x+v_x\cdot t}{\sqrt{4\alpha t}}\bigg),
\label{eqn:addiff1}
\end{equation}
where $T_0$ is the constant temperature at $x=0$, $v$ is the groundwater velocity and $\alpha$ is the heat diffusivity coefficient of water. More information can be found e.g. in \cite{HaeSamVoi:92},\cite{Kol:97}.

The mesh for the numerical model consists of 501 nodes combining 500 line elements. The distance between the nodes is $\Delta x=\unit[0.2]{m}$. The boundary conditions applied are as follows:

\begin{itemize}
	\item Left border:
	\begin{compactitem}
		\item constant source term (liquid flow) with $Q=\unit[3.0 \cdot 10^{-7}]{m^3/s}$
		\item constant temperature with $T=\unit[1]{^\circ C}$
	\end{compactitem}
	\item Right border:	
	\begin{compactitem}
		\item constant pressure with $P=\unit[100]{kPa}$
	\end{compactitem}
	\item Initial conditions:
	\begin{compactitem}
		\item pressure with $P=\unit[100]{kPa}$ for whole domain
		\item temperature $T=\unit[0]{^\circ C}$ for whole domain
	\end{compactitem}
	\item Time step:
	\begin{compactitem}
		\item $\Delta t=\unit[133]{s}$
	\end{compactitem}
\end{itemize}
With the given parameters, the \textsc{Neumann} criteria \eqref{eqn:ne-ldh} results on $\operatorname{Ne}=0.5$ which guarantees the numerical stability of the diffusion part of the transport process. The \textit{Courant} criteria, given by
\begin{equation}
	C=\frac{v_x\cdot\Delta t}{\Delta x}\leq1
	\label{eqn:addiff2}
\end{equation}\\
is equal to $C=0.2$.

\subsection{Results}

In Fig.~\ref{fig-addiff-re1} the comparison of analytical and numerical solution is plotted. The figure shows the temperature breakthrough curve at the end of the fracture at $x=\unit[100]{m}$. The numerical results show an acceptable agreement to the analytical solution. In a further step, the diffusion part of the heat transport process was avoided by minimizing the thermal conductivity of the fluid. Fig.~\ref{fig-addiff-re2} shows the breakthrough curve for only advective heat transport.

\begin{figure}[htbp!]
\centering
\includegraphics[width=0.7\textwidth]{PART_II/T/Ad-Diff.eps}
\caption{\label{fig-addiff-re1}Temperature breakthrough curve at the point $x=\unit[100]{m}$.}
\end{figure}

\begin{figure}[htbp!]
\centering
\includegraphics[width=0.7\textwidth]{PART_II/T/advection.eps}
\caption{\label{fig-addiff-re2}Temperature breakthrough curve when diffusion is neglected (shows numerical diffusion)}
\end{figure}
%\clearpage
