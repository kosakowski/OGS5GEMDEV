\section{Porous media mechanics of THM processes}
\label{sec:prob}

We consider heat transport, fluid flow and deformation in fully or
partially saturated porous media. The
governing equations are briefly summarized below.

\subsection{Flow process}
\label{sec:flow}

We consider a general case of a flow problem in deformable porous
media. With the classical Darcy's law, the fluid flux $\Fluxf$ is
defined as
%
\begin{gather}
\Fluxf^{\gamma} = -\poro
\Satc\left(\dfrac{\RelK\per}{\mu^{\gamma}}(\nabla
p^{\gamma}-\densc\grv)\right)
 \label{eq:fflux}
\end{gather}
%
where $\gamma$ specifies gaseous or liquid phase, $\Satc$ is saturation, $p$ is pressure,  $\dens$ is
density, $n$ is effective porosity of porous media, $\mu$ is fluid
viscosity, $\per$ is intrinsic permeability tensor, $\RelK$ is
relative permeability, $\grv$ is gravity acceleration.

The balance equations of fluid phase mass are given by
\begin{gather}
\pD{(\poro\Satc\densc)}{t}+ \nabla\cdot\Fluxf^{\gamma}+\nabla\cdot (
\poro \Satc\densc \dot \Disp)=Q_{\scriptscriptstyle f}^{\gamma}
 \label{eq:gv1}
\end{gather}
for any point $\Point\in \Omega\in \mathbb{R}^{\mathrm{n}}$ with
$\mathrm n$ dimension of the real space. $Q_{\scriptscriptstyle
f}^{\gamma}$ denotes source/sink terms. The unknown field
functions of eqn. (\ref{eq:gv1}) to be solved are fluid phase
saturation $\Satc$ and fluid phase pressure $p^{\gamma}$. Fluid
mass balance is coupled to the deformation of the porous medium
(third term in eqn. \ref{eq:gv1}). $\Disp$ is solid displacement.

With constitutive equations for fluid density, capillary pressure
and  relative permeability, the balance equations (\ref{eq:gv1}) are
closed (\cite{Kol:02}). The boundary conditions for this problem can
be Neumman type as
\begin{gather}
\Fluxf^{\gamma} \cdot \nrl =
q^{\gamma}_{\scriptscriptstyle{\Gamma}}, \forall\, \Point\,\in
\partial \Omega
 \label{eq:bcgv1}
\end{gather}
or Dirichlet  type as
\begin{gather}
\presc = \presc_{\scriptscriptstyle{\Gamma}}, \quad \Satc =
\Satc_{\scriptscriptstyle{\Gamma}},
 \forall\, \Point\, \in \partial \Omega
 \label{eq:bcgv2}
\end{gather}

This type of initial-boundary-value-problem (IBVP) can be solved with
the corresponding initial conditions of unknowns.

\subsection{Deformation process}
\label{sec:deformation}

Assuming solid grains  are incompressible, i.e. $d^s\Disp/d^s t =
0$, deformations in porous media can be described by the momentum
balance equation in the terms of stress as\cite{LewSch82}
\begin{equation}
\nabla \cdot (\Stress -\sum_\gamma^{phase}\sat^\gamma p^\gamma\, \I
) + \dens\grv = 0
\label{eq:momb}
\end{equation}
where $\Stress$ is the effective stress of porous medium, $\I$ is
identity tensor. \rev{In the present study, the traditional
mechanics sign of stress is used.} Density of porous media consists
of the portion contributed by liquid $l$ and by the portion
contributed of solid as
$\dens=\poro\sum_\gamma^{phase}\dens^\gamma+(1-\poro)\dens^s$.
Displacement $\Disp$ is the primary variable to be solved by
substituting the constitutive law for stress-strain behavior
\begin{equation}
\begin{array}{l}
 \Stress = \CT\, \Strain\\
  \Strain = \dfrac{1}{2}(\nabla \Disp+(\nabla \Disp)^{\mathrm T})
 \end{array}
 \label{eq:strstr}
\end{equation}
with $\CT$, a forth order material tensor and $\Strain$, the
strain. Superscript $\mathrm T$ means the transpose of matrix. The
deformation problem can be considered as a boundary value problem
with boundary conditions given by
\begin{equation}
\Stress : \nrl = \bm t
\quad \mbox{or} \quad
\Disp = \Disp_{\scriptscriptstyle{\Gamma}},
\quad \forall\,\Point \in
\partial \Omega
\label{eq:debc}
\end{equation}

\subsection{Heat transport process}
\label{sec:heat}

For heat transport problem, we consider advective and diffusive
fluxes. Heat flux in multi-phase porous media is given by

\begin{equation}
 \Flux_{\mbox{\tiny T}} = -K_e\nabla\,T+\poro\sum_{\gamma}^{phase}(\densc\HC^{\gamma}) T\vel
 \label{eq:tflux}
\end{equation}

where $K_e$ is the heat conductivity, $T$ is temperature,
$\HC^{\gamma}$ is specific heat capacity of fluid phase,
$\vel^\gamma$ is fluid phase velocity.
%
With the definition of heat flux (\ref{eq:tflux}), the governing
equation of heat transport is
\begin{equation}
 \sum_{\gamma}^{phase}(\densc\HC^{\gamma})\pD{T}{t}-\nabla \Flux_{\mbox{\tiny T}}
 =
 Q_{\mbox{\tiny T}}
 \label{eq:tgrn}
\end{equation}
with boundary conditions
\begin{equation}
 \Flux_{\mbox{\tiny T}}\cdot\nrl=\Flux_{\mbox{\tiny
 T}}\vert_{\scriptscriptstyle{\Gamma}},\, \mbox{or}\quad
 T=T_{\scriptscriptstyle{\Gamma}}, \,
 \forall\, \Point\, \in \partial \Omega
 \label{eq:tbc}
\end{equation}
and initial condition
\begin{equation}
 T(\Point)=T_0(\Point), \, \forall\, \Point\, \in \Omega
 \label{eq:tini}
\end{equation}
\\[3mm]
$ Q_{\mbox{\tiny T}}$ are heat sources.
