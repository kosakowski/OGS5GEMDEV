%\section{Conservation principles}

\footnote{Truesduell}

Based on the kinematical foundation (section \ref{sec:kinematics}) we formulate the general conservation principle of continuum mechanics for both Eulerian and Lagrangian points of view (section \ref{sec:euler_lagrange}).
%
The amount of a (conservation) quantity in a defined volume $\Omega$ is given by

\begin{eqnarray}
\Psi =
\int\limits_\Omega \psi d\Omega(t)
\end{eqnarray}

where $\Psi$ is an extensive conservation quantity\index{quantity
- extensive } (i.e. mass, momentum, energy) and $\psi$ is the
corresponding intensive conservation quantity\index{quantity -
intensive} such as mass density $\rho$, momentum density $\rho \bf
v$ or energy density $e$ (see Table
\ref{tab:conservation_quantities}).

%
\begin{table}[htb!]
\caption{Conservation quantities}
\label{tab:conservation_quantities}
\begin{center}
\begin{tabular}{|ll|ll|}
\hline
Extensive  quantity &  Symbol    &  Intensive quantity      &  Symbol  \\
\hline \hline
%\mbox{\rule[1mm]{0cm}{3mm}Mass}
Mass                &  $M$,$M_k$ & Mass density             & $\rho$,$\rho_k$  \\
Linear momentum     &  $\bf m$   & Linear momentum density  & $\rho \bf v$ \\
Energy              &  $E$       & Energy density           & $e = \rho i + \frac 1 2 \rho v^2 $
\\[1pt]
\hline
\end{tabular}
\end{center}
\end{table}
%

The balance equations for mass, momentum and energy conservation can be derived based on two fundamental principles, i.e. Eulerian and Lagrangian frameworks (e.g. \cite{Kol:02})
%
Both conservation principles are related by two different forms of derivatives
\begin{equation}
\frac{d \psi}{dt}
=
\frac{\p \psi}{\p t} + \v\cdot\nabla\psi
\label{eqn:derivatives}
\end{equation}
the total (or material) $d$ and partial derivatives $\p$, respectively.
%
The general integral balance equation is given by
%
\begin{eqnarray}
\frac{d}{dt} \int\limits_\Omega \psi~d\Omega
=
\int\limits_\Omega
\left(
\frac{\partial\psi}{\partial t}
+
\nabla \cdot {\mathbf{\Phi}}
\right)
d\Omega
=
\int\limits_\Omega q^\psi d\Omega
\label{eqn:general_balance_equation}
\end{eqnarray}
%
where $\psi$ is a general conservation quantity, $\mathbf{\Phi}$ is the total flux of $\psi$, and $Q$ is a source/sink term for $\psi$.
%
The corresponding extensive and intensive conservation quantities are summarized in Tab. \ref{tab:conservation_quantities}.

The total flux ${\bf\Phi}^\psi$ of a quantity $\psi$ is defined as
\begin{eqnarray}
{\bf\Phi}^\psi
=
{\bf v}^E \psi
\end{eqnarray}

where ${\bf v}^E$ is a mean particle velocity\index{quantity - velocity -
particle}. Physically ${\bf\Phi}^\psi$ represents the quantity of
$\psi$ passing through a unit area of the continuum, colinear with
${\bf v}^E$, per unit time with respect to a fixed coordinate
syste, i.e. Eulerian point of view.

For the case of a multi-component continuum let ${\bf v}$ denote
the mass-weighted velocity\index{quantity - velocity - mass-weighted}
describing a more ordered motion of the particles of a fluid
element. The total flux can be written as
\begin{eqnarray}
{\bf\Phi}^\psi
=
{\bf v}^E \psi
=
\underbrace{{\bf v} \psi}_{{\bf\Phi}^\psi_A}
+
\underbrace{({\bf v}^E-{\bf v}) \psi}_{{\bf\Phi}^\psi_D}
\end{eqnarray}

and, therefore, decomposed into two parts: an advective flux
${\bf\Phi}^\psi_A$ and a diffusive flux ${\bf\Phi}^\psi_D$
relative to the mass-weighted velocity:

\begin{itemize}
 \item
Advective flux\index{flux - advective} of quantity $\psi$
\begin{eqnarray}
{\bf\Phi}^\psi_A
=
{\bf v}\psi
\end{eqnarray}

 \item
Diffusive flux\index{flux - diffusive} of quantity $\psi$
(Fick's law)\index{law - Fick}
\begin{eqnarray}
{\bf\Phi}^\psi_D=
-\alpha \nabla\psi
\end{eqnarray}
\end{itemize}

where $\alpha$ is a diffusivity coefficient. The negative sign indicates, that diffusive flux is positive in the direction of negative gradient.

If the conservation quantity is a vector (e.g. linear momentum)
then the flux becomes a tensor and the source term a vector (e.g. body forces):

\begin{itemize}

\item Advective flux of vector quantity $\psi$
%
\begin{eqnarray}
{\bf\Phi}^\psi_A
=
% {\bf v}:{\Boldmath\psi}
{\bf v}:{\mathbf{\psi}}
=
\left[
\begin{array}{ccc}
v_x & v_y & v_z
\end{array}
\right]
\left[
\begin{array}{c}
\psi_x \\ \psi_y \\ \psi_z
\end{array}
\right]
=
\left|
\begin{array} {lll}
v_x \psi_x &  v_x \psi_y & v_x \psi_z \\
v_y \psi_x &  v_y \psi_y & v_y \psi_z \\
v_z \psi_x &  v_z \psi_y & v_z \psi_z
\end{array}
\right|
\end{eqnarray}

\item Diffusive flux of vector quantity $\psi$
%
\begin{eqnarray}
{\bf\Phi}^\psi_D
=
% -\rho \nabla:{\Boldmath\psi}
-\rho \nabla:{\mathbf{\psi}}
=
-\alpha
\left|
\begin{array} {lll}
\frac{\partial\psi_x}{\partial x} & \frac{\partial\psi_y}{\partial y} & \frac{\partial\psi_z}{\partial z} \\
\frac{\partial\psi_x}{\partial x} & \frac{\partial\psi_y}{\partial y} & \frac{\partial\psi_z}{\partial z} \\
\frac{\partial\psi_x}{\partial x} & \frac{\partial\psi_y}{\partial y} & \frac{\partial\psi_z}{\partial z}
\end{array}
\right|
\end{eqnarray}
\end{itemize}

When substituting the flux definition into the general balance equation (\ref{eqn:general_balance_equation}),
we yield the so-called transport equation

\begin{eqnarray}
\frac{d}{dt} \int\limits_\Omega \psi~d\Omega
=
\underbrace{\int\limits_\Omega \frac{\partial\psi}{\partial t}~d\Omega}_{1}
+
\underbrace{\int\limits_\Omega \nabla \cdot ({\bf v}\psi)~d\Omega}_{2}
-
\underbrace{\int\limits_\Omega \nabla \cdot (\alpha\nabla\psi)~d\Omega}_{3}
=
\underbrace{\int\limits_\Omega q^\psi}_{4}
\label{eqn:integral_balance_equation}
\end{eqnarray}

with the followind terms:

\renewcommand{\arraystretch}{0.9}
\begin{enumerate}
 \item Rate of increase of $\psi$ within a fluid element
 \item Net rate of $\psi$ due to flux out of the fluid element
 \item Rate of increase / decrease of $\psi$ due to diffusion
 \item Rate of increase / decrease of $\psi$ due to sources
\end{enumerate}
