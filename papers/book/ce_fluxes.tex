\section{Constitutive equations - Fluxes}
\label{sec:constitutive_equations_fluxes}

In general, two types of fluxes have to be
considered: advective fluxes and diffusive fluxes of fluid mass or energy.

%-------------------------------------------------------------------------
\subsection{Fluid flux - Darcy's law}

The relationship for fluid fluxes in porous media (\ref{eqn:momentum_balance_fluid}), i.e. Darcy's law, can be also considered as a phenomenological equation.
This means the fluid flux is assumed to be proportional to the pressure gradient and the gravity force.

\begin{eqnarray}
\Flux^{\Phase s}
=
\Porosity \Saturation^\Phase (\VelocityVector^{\gamma} - \SolidVelocityVector)
=
-
\Porosity \Saturation^\Phase
\left(
\frac{\PermRelP^{\gamma}\PermTensor}{\Viscosity^\gamma}
(
\nabla\Pressure^\gamma
-
\Density^\gamma \GravityVector
)
\right)
\end{eqnarray}

%-------------------------------------------------------------------------
\subsection{Heat flux - Fourier's law}

Total diffusive heat flux in the porous medium
is given by Fourier's law for heat conduction
%
\begin{eqnarray}
\HeatFluxDiffusive
&=&
-
\ThermalConductivity\nabla\Temperature
\label{energy flux conductive}
\end{eqnarray}
%
where $\ThermalConductivity$ is the thermal conductivity of the porous medium.

%-------------------------------------------------------------------------
\subsection{Mass flux - Fick's law}

Diffusive mass flux of components within the fluid phase
is given by Fick's law driven by mass fractions.
%
\begin{eqnarray}
\FluxDiffusive_k^\Phase
&=&
\Porosity
\Saturation^\Phase
\Density_k^\Phase
(\VelocityVector_k^g - \VelocityVector^g)
\nonumber\\
%---
&=&
-
\Porosity
\Saturation^\Phase
\Density^\Phase
D_k^\Phase
\nabla\MassFraction_k^\Phase
\end{eqnarray}
%
where
$D_k^\Phase$ is the multicomponental diffusion coefficient of component $k$ in phase $\Phase$.
Note, that due to balance requirements it has to be:
$\FluxDiffusive_a^\Phase+\FluxDiffusive_w^\Phase=0$.
This relationship ensures that the total diffusive mass flux in a fluid phase
summed over all components is zero with respect to the mass average velocity (Falta et el. 1992).