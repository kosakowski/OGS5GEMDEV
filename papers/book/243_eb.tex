\subsection{Energy balance}

\subsubsection{Heat transport}

The equation of energy conservation is derived from the first law
of thermodynamics\index{law - thermodynamics} which states that
the variation of total energy of a system is due to the
work\index{quantity - work} of acting forces and heat transmitted to the
system.

The total energy\index{quantity - energy - total} per unit mass $e$ specific
energy) can be defined as the sum of internal (thermal)
energy\index{quantity - energy - internal} $i$ and specific kinetic
energy\index{quantity - energy - kinetic} $v^2/2$. Internal energy is due to
molecular movement. Gravitation is considered as an energy source
term, i.e. a body force which does work on the fluid element as it
moves through the gravity field. 
%
The conservation quantity for energy balance is total energy density
%
\begin{eqnarray}
\psi^e = \rho e = \rho(i+v^2/2)
\end{eqnarray}

Using mass and momentum conservation we can derive the following balance equation for the internal energy.
%
\begin{eqnarray}
\rho\frac{di}{dt}
=
\rho q^i
-
\nabla\cdot\mathbf{j}_{\mathrm{th}}
+
\sigma\cdot\nabla\v
\end{eqnarray}
%
where 
$q^i$ is the internal energy (heat) source, 
$\mathbf{j}_{\mathrm{th}}$ is the diffusive heat flux.
%
Applying the chain to the left hand side of the above equation yields
%
\begin{eqnarray}
\rho\frac{di}{dt}
=
\rho\frac{d cT}{dt}
=
\rho c\frac{d T}{dt}
+
\rho T \frac{d c}{dt}
\end{eqnarray}
%
as well as the definition of the material derivative
%
\begin{eqnarray}
\frac{d T}{dt}
=
\frac{\p T}{\p t}
+
\v\cdot\nabla T
\end{eqnarray}
%
we obtain the heat energy balance equation
%
\begin{eqnarray}
\rho c \frac{\p T}{\p t}
+
\rho c \v\cdot\nabla T
-
\nabla\cdot\lambda\nabla T
+
\rho T \frac{dc}{dt}
-
\sigma\cdot\nabla\v
= 
\rho q_{\mathrm{th}}
\end{eqnarray}

\subsubsection{Porous medium}

The heat balance equation for the porous medium consisting of several solid and fluid phases
is given by

\fbox{
\parbox{12cm}
{
\begin{eqnarray}
( \sum_\alpha \epsilon^\alpha c^\alpha \rho^\alpha )
\frac{\p T}{\p t}
+
\nabla\cdot
\left(
(\sum_\gamma n S^\gamma \rho^\gamma c^\gamma \mio{v}{\gamma}{}{}) T 
- 
( \sum_\alpha \epsilon^\alpha \lambda^\alpha)
\nabla T
\right)
= 
\nonumber \\
\sum_\alpha \epsilon^\alpha \rho^\alpha q_{\mathrm{th}}
+ 
( \sum_\gamma n S^\gamma \mio{v}{\gamma}{}{} ) \cdot \nabla \mathbf\sigma
\label{eqn:energy_balance}
\end{eqnarray}
}}
%
where $\alpha$ is all phases and $\gamma$ is fluid phases, respectively.

%
Most important is the assumption of local thermodynamic equilibrium, meaning that all phase temperatures are equal and , therefore, phase contributions can be superposed.
The phase change terms cancel out with the addition of the individual phases. 
