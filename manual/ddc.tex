\section{Domain Decomposition}

\begin{tabular*}{5.35cm}{|p{2.5cm}|p{2cm}|} \hline
Object acronym & DDC \\
C++ class  & CPARDomain \\
Source files   & par\_ddc.h/cpp \\
\hline
File extension & *.ddc\\
Object keyword &  {\texttt{\#DOMAIN}} \\
\hline
\end{tabular*}


%-------------------------------------------------------------------------------
\subsection{\bf\texttt{\#DOMAIN}}

\begin{verbatim}
#DOMAIN
 $ELEMENTS
 element_numbers
 $NODES_INNER
 inner_node_numbers
 $NODES_HALO
 halo_node_numbers
\end{verbatim}

\begin{tabular*}{12.773cm}{|p{3.cm}|p{8.9cm}|} \hline
Subkeyword         & Meaning \\ \hline \hline
%
\texttt{ELEMENTS}    & Element numbers of this domain \\
\texttt{NODES\_INNER}& Numbers of inner nodes \\
\texttt{NODES\_BORDER} & Numbers of domain boundary nodes \\
\hline
\end{tabular*}

%-------------------------------------------------------------------------------

\Examples{
%-------------------------------------------------------------------------------
\subsection{Examples}
%-------------------------------------------------------------------------------
\subsubsection{(confined) Groundwater flow}
\begin{verbatim}
benchmarks: h_tri.ddc
#DOMAIN
 $ELEMENTS
 0
 1
 2
 3
 8
 9
 10
 11
 16
 17
 18
 19
 $NODES_INNER
 0
 1
 5
 6
 10
 11
 15
 16
 $NODES_BORDER
 2
 7
 12
 17
#STOP
\end{verbatim}
}
%%WW
When high order interpolation is required,  additional element nodes except vertex node will be intrdoduced.
Under such situdation, nodes on so called borders, nodes on interfaces between adjecent domains,
 will be computed in the programm. Therefore, as an alternative, all nodes of a domain can be listed
 after keyword \mbox{\$NODES\_INNER} as
\begin{verbatim}
benchmarks: h_tri.ddc#DOMAIN
 $ELEMENTS
 0
 1
 2
 3
 8
 9
 10
 11
 16
 17
 18
 19
 $NODES_INNER
 0
 1
 5
 6
 10
 11
 15
 16
 2
 7
 12
 17
#STOP
\end{verbatim}

\LastModified{WW \today}
\newpage
