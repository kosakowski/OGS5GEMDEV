\section{Solver}
Solving linear equation systems arising from the finite element discretizations is a crucial and the most time consuming
point in finite element analysis of PDE. Rockflow provides several most popular and efficient linear solvers.
Nonlinear PDEs results in corresponding nonlinear equation system, which have to be linearized. Nonlinear problems of interest are: Forchheimer flow, gas flow, multi-phase flow, density driven flow, reactive
transport, nonlinear deformations and so on. Two nonlinear solvers, methods to linearize the non-linear equation system,
are available in Rockflow. They are Picard fix point method and Newton-Raphson method.

\subsection{Nonlinear solver}
The following keywords can be used to choose the nonlinear solver for corresponding problem:
\begin{center}
  \begin{verbatim}
   #NONLINEAR_SOLVER_PROPERTIES_PRESSURE
   #NONLINEAR_SOLVER_PROPERTIES_WATERCONTENT
   #NONLINEAR_SOLVER_PROPERTIES_SATURATION
   #NONLINEAR_SOLVER_PROPERTIES_CONCENTRATION
   #NONLINEAR_SOLVER_PROPERTIES_SOLUTE_CONCENTRATION
   #NONLINEAR_SOLVER_PROPERTIES_SORBED_CONCENTRATION
   #NONLINEAR_SOLVER_PROPERTIES_TEMPERATURE_PHASE
 \end{verbatim}
\end{center}

Data specifying the property of the chosen solve have to be followed after the corresponding key word and separated
by blank space. The data appear in the order as


\begin{itemize}
    \item Solver type, \emph {integer},\\
       1 for Picard fixed point ieration\\
       2 for Newton-Raphson method
    \item Maximum number of iteration,  \emph {integer}
    \item Type of error calculation, integer,\\
        1 for for difference in result vector\\
        2 for difference in residuum
    \item Absolute error tolerance,  \emph {float}
    \item Relative error tolerance,  \emph {float}
    \item Combined error tolerance with respect to linear iterative solver,  \emph {float}
    \item Number of iterations before reassembly of the system equations,  \emph {float}
    \item Flag to show time consumed,  \emph {integer}
    \item Type of showing time consumed,  \emph {integer}
\end{itemize}

Following is a sample of defining a Newtion-Raphson solver for pressure analysis:\\

%\begin{center}
\shadowbox
 {
   \parbox[c]{10cm}
   {

         \#NONLINEAR\_SOLVER\_PROPERTIES\_PRESSURE\\
         2 100 1 1.e06 1.e3 0.0 1 -1 0
   }
  }
%\end{center}

\subsection{Linear solver}
Direct and iterative solvers are listed in the following box with their indeces.\\
%\begin{center}
 \begin{tabular}{||lllll||}
  \hline
  \hline
   1: SpGAUSS & 2:SpBICGSTAB &3:SpBICG     & 4:SpQMRCGSTAB & 5: SpCG\\
   6: SpCGNR  & 7:CGS        &8: SpRichard & 9:SpJOR       & 10:SpSOR\\
  \hline
  \hline
  \end{tabular}\\
%\end{center}
Only the first one is direct solver.

Using pre-conditioners, which is an approximate solution of the defect equation of the linear equation solver, is very
helpful to accelerate the convergence of the iterative solver. There are three kinds of pre-conditioners available in
Rockflow such as Jacobi, incomplete LU decomposition (ILU) and SOR splitting  of system matrix method.


The following keywords can be used to choose the linear solver for corresponding problem:
%\begin{center}
  \begin{verbatim}
    #LINEAR_SOLVER_PROPERTIES_PRESSURE
    #LINEAR_SOLVER_PROPERTIES_WATER_CONTENT
    #LINEAR_SOLVER_PROPERTIES_SATURATION
    #LINEAR_SOLVER_PROPERTIES_CONCENTRATION
    #LINEAR_SOLVER_PROPERTIES_SORBED_CONCENTRATION
    #LINEAR_SOLVER_PROPERTIES_SOLUTE_CONCENTRATION
    #LINEAR_SOLVER_PROPERTIES_TEMPERATURE_PHASE
    #LINEAR_SOLVER_PROPERTIES_DISPLACEMENT
    #LINEAR_SOLVER_PROPERTIES_IMMOBILE_SOLUTE_CONCENTRATION
 \end{verbatim}
%\end{center}

The data of solver property following the keyword is in the order as

{\ttfamily
  \small
\begin{tabular}{|l|c|l|l|}
  \hline
  Position &Value & Data type & Meaning \\
  \hline
  \hline % 1st row

    1&   &Integer  &Type of solver:\\
    & 1 &         &Gaussian (direct solver)\\
    & 2 &         & BiCGSTAB (default for transport)\\
    & 3 &         & BICG                             \\
    & 4 &         & QMRCGSTAB \\
    & 5 &         & CG  (default for flow analysis)\\
    & 6 &         & CGNR  \\
    & 7 &         & CGS   \\
    & 8 &         & Richard \\
    & 9 &         & JOR    \\
    & 10 &         & SOR  \\
  \hline % 2nd row
   2&   &Integer &Norm of error:\\
    & 0 &         &Maximum norm\\
    & 1 &         &Unity norm\\
    & 2 &         &Eucleadian norm  \\
  \hline % 3rd row
   3&   &Integer &Type of pre-conditioners\\
    & 0 &      &No pre-conditioner   \\
    & 1 &      &Jacobi    \\
    & 10&      &SSOR    \\
    &10 &      & ILU    \\
  \hline % 4th row
   4&   &      &Maximum number of solver iterations\\
  \hline % 5th row
   5&   &      &Number of repeated solving\\
  \hline % 6th row
   6&   & Integer   &Method of error calculation\\
    &  0 &    &Absolute error\\
    &  1 &    &Relative error with respect to the RHS vector $\| b_0\|$\\
    &  2 &    &Relative error with respect to the residuum $\| r_0\|$\\
    &  3 &    &variable error (if $\| r_0\|<1$) then 0 else 2 \\
    &  4 &    &Relative error with respect to actual residuum $\| r_0\|/\| x\|$\\
 \hline % 7th row
\end{tabular}\\
\begin{tabular}{|l|c|l|l|}
\hline % 7th row
   7\phantom{osition} &\phantom{Value} & Float\phantom{type}  &Tolerance of iteration\phantom{spect to actual residuum $\| r_0\|/\| x\|$}\\
 \hline % 8th row
   8& 0  & Integer   & \\
 \hline % 9th row
   9&    & Integer  & Validity time interval if BiCGSTAB is chosen\\
    &    & Integer  & Matrix storage method: \\
    &  1  &   & Full matrix \\
    &  2  &   & Sparse matrix \\
    &  4  &   & Unsymmetrical sparse matrix\\
 \hline % 10th row
\end{tabular}
}


Here is a sample of specifying BiCGSTAB solver for deformation analysis with Jacobi pre-conditioner
and symmetrical sparse matrix storage:\\

%\begin{center}
\shadowbox
 {
   \parbox[c]{10cm}
   {
     \#LINEAR\_SOLVER\_PROPERTIES\_DISPLACEMENT\\
     2   0  1  10000  0  0  1.000000e-9  0  2

   }
  }
%\end{center}
