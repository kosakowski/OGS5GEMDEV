\section{Remarks}

\subsection{V4.1.**MX}

Based on the current version, the following points are to be
considered in the further developments of FEM:

1. Time steps: For reactive transport, different time steps are
needed for different processes. The realized function for that in
the current version is valid for TH2C coupled processes. But for
more processes, a more robust method should be developed in time
class.

2. Various initial conditions for different material groups: This
is the problem of DECOVALEX. For THC simulation,
$\$Material\_DOMAIN$ is used. This can be used only if the mesh is
already generated in *.rfi. This should be in a general way
according to mesh type like $SURFACE$ for 2D.

3. Reaction class: there are two classes (REACT and
REACTION\_MODEL) for reaction. Join them into one?

\subsection{V4.1.13OK}
\subsection{V4.2.04WW}
Gravity orientation is automatically determined by the coordinates 
given  in mesh file as: 
\begin{itemize}
  \item 1D: Only $x$ component has non-zero number $\longrightarrow x$  direction, e.g.,   
           \begin{itemize}
            \item   [x1] 0.0 0.0
            \item   [x2] 0.0 0.0   
            \item  $\vdots$
           \end{itemize}
              
  \item 1D:  Only $z$ component has non-zero number $\longrightarrow z$  direction, e.g.,  
            \begin{itemize}
            \item   0.0 0.0 [z1]
            \item   0.0 0.0 [z2]  
            \item  $\vdots$
           \end{itemize}
  \item 2D:  Gravity in $z$ orientation if $x$ and $z$ components has non-zero number,  
            \begin{itemize}
            \item  [x1] 0.0 [z1]
            \item  [x2] 0.0 [z2]  
            \item  $\vdots$
           \end{itemize}
  \item 2D:  Gravity in $y$ orientation if $x$ and $y$ components has non-zero number,  
            \begin{itemize}
            \item  [x1] [y1]  0.0
            \item  [x2] [y2]  0.0 
            \item  $\vdots$
            \end{itemize}

 \item 3D:  Gravity in always in $z$ orientation even if fracture network is involved.  
\end{itemize}




ToDo things
\begin{itemize}
    \item H-Richards - Celia (YD)
    \item H-Richards - Quad (YD)
    \item M (PCS keyword) (WW)
    \item HM (PCS keyword) (WW)
    \item THM
    \item TH2M
\end{itemize}

\LastModified{OK \today}
