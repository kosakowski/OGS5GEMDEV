\section{Initial Conditions}

\begin{tabular*}{8.35cm}{|p{2.5cm}|p{5cm}|} \hline
Object acronym & IC \\
C++ class      & CInitialCondition \\
Source files   & rf\_ic\_new.h/cpp \\
\hline
File extension & *.ic \\
Object keyword & \#INITIAL\_CONDITION \\
\hline
\end{tabular*}

\Developer{
%----------------------------------------------------------
\subsection{Theory}
}

%-------------------------------------------------------------------------------
\subsection{\texttt{\bf\#INITIAL\_CONDITION}}

%\subsubsection{Keyword structure}

\begin{verbatim}
#INITIAL_CONDITION
 $PCS_TYPE // physical process
  LIQUID_FLOW       // H process (incompressible flow)
  UNCONFINED_FLOW   // H process (incompressible flow)
  GAS_FLOW          // H process (compressible flow)
  TWO_PHASE_FLOW    // H2 process (incompressible/compressible flow)
  COMPONENTAL_FLOW  // H2 process (incompressible/compressible flow)
  RIVER_FLOW        // H process (incompressible flow)
  RICHARDS_FLOW     // H process (incompressible flow)
  OVERLAND_FLOW     // H process (incompressible flow)
  HEAT_TRANSPORT    // T process (single/multi-phase flow)
  DEFORMATION       // M process (single/multi-phase flow)
  MASS_TRANSPORT    // C process (single/multi-phase flow)
  GROUNDWATER_FLOW  // H process (incompressible flow)
 $PRIMARY_VARIABLE
  PRESSURE1      // flow (phase)
  SATURATION2
  TEMPERATURE1   // heat transport
  DISPLACEMENT_X // deformation
  DISPLACEMENT_Y
  DISPLACEMENT_Z
  CONCENTRATION1 // mass transport
  CONCENTRATIONx
 $GEO_TYPE // geometry
  POINT    name
  POLYLINE name
  SURFACE  name
  VOLUME   name
  DOMAIN
  SUB_DOMAIN
  MATERIAL_DOMAIN number (material group defined in *.rfi)
 $DIS_TYPE // value distribution
  CONSTANT     value
  DISTRIBUTED  file_name
  GRADIENT     value1, value2, value3
  GRADIENT_Z   value1, value2, value3
  RESTART      file_name.rfr
\end{verbatim}


\begin{tabular*}{12.773cm}{|p{3.cm}|p{1.5cm}|p{7cm}|} \hline
Parameter          & Acronym & Meaning \\ \hline \hline
%
\texttt{PCS\_TYPE} & PCS &  Reference to a process \\
\texttt{GEO\_TYPE} & GEO &  Reference to a geometric object \\
\texttt{DIS\_TYPE} & DIS &  Distribution of values \\
\hline
\end{tabular*}

\subsubsection{\texttt{\$PCS\_TYPE}}

\begin{tabular*}{12.773cm}{|p{3.cm}|p{1.5cm}|p{7cm}|} \hline
Parameter          & Value & Meaning \\ \hline \hline
%
\texttt{PRESSUREx}        & phase & Source term for fluid phase x \\
\texttt{DISPLACEMENTx\_X} & phase & Load force for solid phase x \\
\texttt{DISPLACEMENTx\_Y} & phase & ... \\
\texttt{DISPLACEMENTx\_Z} & phase & ... \\
\texttt{TEMPERATUREx}     & phase & Source term for temperature \\
\texttt{CONCENTRATIONx}   & component & Source term for component mass \\
\hline
\end{tabular*}

\subsubsection{\texttt{\$GEO\_TYPE}}

\begin{tabular*}{12.773cm}{|p{3.cm}|p{8.9cm}|} \hline
Parameter          & Meaning \\ \hline \hline
%
\texttt{POINT}     & Name of point \\
\texttt{POLYLINE}  & Name of polyline \\
\texttt{SURFACE}   & Name of surface \\
\texttt{VOLUME }   & Name of volume \\
\texttt{DOMAIN }   & whole domain \\
\texttt{SUB\_DOMAIN }  & patches of domain \\
\texttt{MATERIAL\_DOMAIN }  & Number of material group \\
\hline
\end{tabular*}

%%WW
Initialization of specified zones in a domain can be done by using
 \texttt{SUB\_DOMAIN} for keyword \texttt{\$GEO\_TYPE} followed data
 as the number of sub-domains, pair data for indeces of material
 type (given in .rfi file) and initial values, i.e,
 \begin{verbatim}
 #INITIAL_CONDITION
 $PCS_TYPE
  HEAT_TRANSPORT
 $PRIMARY_VARIABLE
  TEMPERATURE1
 $GEO_TYPE
  SUB_DOMAIN
   2
   0  25.0
   1  23.0
 $DIS_TYPE
  GRADIENT   500.0  10.0  0.03
#STOP
\end{verbatim}
Where number 2 after keyword "SUB\_DOMAIN" indicates there are two set of initial values. For each pair of data,
 the integer defines patch index, which is given in the file about finite element mesh data; the float one gives
 value.  After reading these initial data, the programme will search the elements, which have the same patch
  indices given in "SUB\_DOMAIN" section, and assign the corresponding values.
\begin{description}
  \item[Initial stress by defining a linear function]
   For stress variables in deformation analysis, we can give initial values of stresses by a linear expression such as
 \begin{verbatim}
#INITIAL_CONDITION
 $PCS_TYPE
  DEFORMATION
 $PRIMARY_VARIABLE
  STRESS_XX
 $GEO_TYPE
  SUB_DOMAIN
   4
   0   -23.75+-0.2*y
   1   24.75+-0.5*y
   2   26.75+12.*y+-10*x
   3   27.75+14.0*y+-10*x
#INITIAL_CONDITION
 $PCS_TYPE
  DEFORMATION
 $PRIMARY_VARIABLE
  STRESS_YY
 $GEO_TYPE
  SUB_DOMAIN
  4
  0   23.75+0.2*y
   1   24.75+1.3*y
   2   26.75+16.*y+-20*x
   3   27.75+-18.0*y+-20*x
#STOP
\end{verbatim}
Such an expression takes form
\[a+b*x+c*y+d*z\] with $a,b,c,d$ the float numbers. Note: There is no blank space in the expression.
Operator $+$ is used as both delimiter and operator in the expression. If coefficients $b,c$ or $d$ is negative,
symbol $+$ must be given before the coefficient, e.g. 27.75+-18.0*y+-20*x with $b=-20.0$ and $c=18.0$.
\end{description}

\subsubsection{\texttt{\$DIS\_TYPE}}

\begin{tabular*}{12.773cm}{|p{3.cm}|p{8.9cm}|} \hline
Parameter          & Meaning \\ \hline \hline
%
\texttt{CONSTANT}  & constant value \\
\texttt{GRADIENT}  & ref position, ref value, gradient \\
\texttt{GRADIENT\_Z}  & ref position (z), ref value, gradient \\
\hline
\end{tabular*}

\Examples{
%-------------------------------------------------------------------------------
\subsection{Examples}

%-------------------------------------------------------------------------------
\subsubsection{Initial condition in the domain}

\begin{verbatim}
benchmark: h_line.ic
#INITIAL_CONDITION
 $PCS_TYPE
  LIQUID_FLOW
 $PRIMARY_VARIABLE
  PRESSURE1
 $GEO_TYPE
  DOMAIN
 $DIS_TYPE
  CONSTANT 0.0
#STOP
\end{verbatim}

This initial condition is defined for a process with primary
variable \texttt{PRESSURE1}. Geometrically the initial condition
is assigned to the whole domain, i.e. all mesh points.

Here the use of gradient applies temperature values according to
depth z inside the domain specified. The first value gives the
depth of a reference point, the second value its temperature in
this case, the third value the change in temperature with depth.

\begin{verbatim}
 #INITIAL_CONDITION
 $PCS_TYPE
  TEMPERATURE1
 $GEO_TYPE
  DOMAIN
 $DIS_TYPE
  GRADIENT_Z   500.0  10.0  0.03
#STOP
\end{verbatim}

%-------------------------------------------------------------------------------
\subsubsection{Restart as initial conditions}

Initial conditions can be used for restarts. A corresponding RFR
file (restart.rfr) is required. We present an example for
unsaturated Richards flow.

\begin{verbatim}
#INITIAL_CONDITION
 $PCS_TYPE
  RICHARDS_FLOW
 $PRIMARY_VARIABLE
  PRESSURE1
 $GEO_TYPE
  DOMAIN
 $DIS_TYPE
  RESTART restart.rfr
\end{verbatim}

Example for an RFR file:

\begin{verbatim}
#0#0#0#1#3.236234700193E-04#0#4.2.18# // not used
1   1   4                             // not used
1   1                                 // number of node variables (first number)
PRESSURE1,  Pa                        // name of node variable, unit
0   0.000000000000000e+000            // node number, node value
1  -1.941899651235000e+001
2  -3.880014828104000e+001
3  -5.814051970137000e+001
4  -7.744430939151999e+001
...
231 -3.245527412767000e+002
232 -3.245527412767000e+002
\end{verbatim}

Information of restart time comes from the TIM data (.tim file).
Important is that number of node variables and variable names are
compatible. In the data block node values for each mesh node are
required.

}
 \LastModified{YD - 5th July 2006}
