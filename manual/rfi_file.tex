\section{MSHLib Data}

\subsection{RFI Data}

This is an old file format, which will be no more developed and
supported.

\begin{verbatim}
filename.rfi
\end{verbatim}

The RFI file is finite element mesh (topologic) information.
It is the input file for node and element data. The RFI file
contains 3 blocks: data control, data area for nodes and data area
for elements.

\subsubsection{Example}

\small
\begin{verbatim}
// Data control - this line is not to put in the RFI file
#0#0#0#1#0.0#0##################################################
 0   1733   3296
// Data block for nodes  - this line is not to put in the RFI file
 0    4075.3790000000    1237.6610000000   -10000.000000000
 1    3248.3140000000    1550.1820000000   -8799.0780000000
 2    134.19960000000    1216.1690000000   -5316.9770000000
 3   -3434.3590000000   -1243.2360000000   -2755.8600000000
 4    3348.2520000000   -21.761480000000   -10000.000000000
 5   -2412.6870000000   -10000.000000000   -10000.000000000
 6   -10000.000000000   -10000.000000000   -956.06250000000
 7   -10000.000000000   -8610.7600000000    0.00020700000000000
 8   -6014.6410000000   -1707.9160000000    0.00000000000000
 9   -2160.1140000000    970.08530000000   -2751.5490000000
 ...
 1730   -2557.9790848525    2911.3944854435   -941.30925539516
 1731   -1408.0551330575    2982.2914976413   -2263.2068487507
 1732   -3228.0124258355    2264.7447395733   -587.66012303999
// Data block for elements - this line is not to put in the RFI file
 0   0  line  321    60
 1   4  tri    58   328    59
 2   5  quad  304     1   351    17
 3   2  tet   339   294   337   219
 4   1  pris  328   311   297    54    53   307
 5   3  hex   344   314   343    28   319    29   307   316
 6   0  line_radial 321    60              // radial-symmetric element
 7   4  tri_axial    58   328    59        // axial-symmetric element
 8   5  quad_axial  304     1   351    17  // axial-symmetric element
 ...
 3295   6  tri  1729   1732   1711
\end{verbatim}
\normalsize

\newpage
%------------------------------------------------------------------
\subsubsection{Data control area}

The parameters of the head line are separated by double-crosses.

\small
\hrule
\begin{minipage}[t]{4cm}
Variable name
\begin{verbatim}
art
bin
nr
geom
startzeit
zeitschritt
rfi_filetype
\end{verbatim}
\end{minipage}
%
\begin{minipage}[t]{4cm}
Parameter meaning
\begin{verbatim}
0: has to be zero
0: ASCII format
0: file number
1: output of geometry
0.0: start time of the simulation
0: time step number of previous simulation
3831: RF/RM version number // alternatively
\end{verbatim}
\end{minipage}
\hrule
\normalsize

\bigskip
The parameters of the second line are as follows.

\small
\hrule
\begin{minipage}[t]{4cm}
Variable name
\begin{verbatim}
d
anz_n
anz_e
\end{verbatim}
\end{minipage}
%
\begin{minipage}[t]{4cm}
Parameter meaning
\begin{verbatim}
0: file type
1733: number of nodes
3296: number of elements
\end{verbatim}
\end{minipage}
\hrule
\normalsize


%------------------------------------------------------------------
\subsubsection{Node data}

Geometric node data are: node number (has to start with 0) and node
coordinates.

\small
\hrule
\begin{minipage}[t]{4cm}
\begin{verbatim}
node number | x                  | y               | z

       1732   -3228.0124258355    2264.7447395733   -587.66012303999

\end{verbatim}
\end{minipage}
\hrule
\normalsize

%------------------------------------------------------------------
\subsubsection{Element data}

Topologic element data are: element number (has to start with 0),
material group, geometric element type and element nodes. Number
of nodes per element depends on the geometric element type.

\begin{center}
\begin{tabular}{|ll|l|l|}
\hline
Element type  & etyp & Name & Number of nodes \\
\hline \hline
%
Line          & 1    & line & 2 \\
Quadrilateral & 2    & quad & 4 \\
Hexahedron    & 3    & hex  & 8 \\
Triangle      & 4    & tri  & 3 \\
Tetrahedron   & 5    & tet  & 4 \\
Prism         & 6    & pris & 6 \\
\hline
\end{tabular}
\end{center}


\small
\hrule
\begin{minipage}[t]{4cm}
\begin{verbatim}
element number | material group | element type | element nodes

 0               0                line           321    60
 ...
 3295            6                tri            1729   1732   1711

\end{verbatim}
\end{minipage}
\hrule \normalsize


%%%%%%%%%%%%%%%%%%%%%%%%%%%%%%%%%%%%%%%%%%%%%%%%%%%%%%%%%%%%%%%%%%%%%%%%%%
\newpage
\subsection{MultiMSH Data}
\label{sec:msh}

since 4.1.15

\begin{tabular*}{5.85cm}{|p{2.5cm}|p{2.5cm}|} \hline
Object acronym & MSH \\
C++ class  & CFEMesh \\
Source files   & rf\_ele\_msh.h/cpp \\
\hline
File extension & *.msh\\
Object keyword &  {\texttt{\#FEM\_MSH}} \\
\hline
\end{tabular*}

Each process can be provided with a different mesh by means of the
\texttt{\$PCS\_TYPE} subkeyword. If only one mesh is provided it
is used for all processes defined.

%-------------------------------------------------------------------------
\subsubsection{Example}

\small
\begin{verbatim}
GeoSys-MSH: Mesh ------------------------------------------------
#FEM_MSH
 $PCS_TYPE
  GROUNDWATER_FLOW
 $GEO_TYPE //4.3.20
  geo_type_name geo_type
 $NODES
  808
  0 0.000000000000e+000 1.000000000000e+000 -5.000000000000e+000
  ...
 $ELEMENTS
  300
  0 1 -1 hex 0 4 103 3 202 206 305 205
 $LAYER
  5
 $PCS_TYPE
  RICHARDS_FLOW
  ...
 $PCS_TYPE
  OVERLAND_FLOW
  ...
#STOP
\end{verbatim}
\normalsize

%WW
If keyword \texttt{\$AXISYMMETRY} is used, the mesh data are
ready for the simulation of an axisymmetrical problem with
coordinate (r, z, 0.0) of each node. More information of axisymmetry
is given in section \ref{sec:pcs}.

%OK
\small
\begin{verbatim}
$GEO_TYPE //4.3.20
\end{verbatim}
\normalsize \vspace{-2mm}
%
This subkeyword describes a MSH-GEO relationship. It can be used
e.g. for a regional soil model. The MSH contains numerous soil
columns, which are treated as several local soil problems.


%-------------------------------------------------------------------------
\subsubsection{MSH-PCS relation}

\small
\begin{verbatim}
 $PCS_TYPE
  GROUNDWATER_FLOW
 ...
 $PCS_TYPE
  RICHARDS_FLOW
  ...
 $PCS_TYPE
  OVERLAND_FLOW
  ...
\end{verbatim}
\normalsize

%-------------------------------------------------------------------------
\subsubsection{MSH node data}

\small
\begin{verbatim}
 $NODES
  808
  0 0.000000000000e+000 1.000000000000e+000 -5.000000000000e+000
  ...
\end{verbatim}
\normalsize

%-------------------------------------------------------------------------
\subsubsection{MSH element data}

\small
\begin{verbatim}
 $ELEMENTS
  300
  0 1 -1 hex 0 4 103 3 202 206 305 205
  ...
\end{verbatim}
\normalsize

%-------------------------------------------------------------------------
\subsubsection{MSH AddOn data}

\small
\begin{verbatim}
 $LAYER
  5
\end{verbatim}
\normalsize

\vfill \LastModified{OK 28.06.2005}
