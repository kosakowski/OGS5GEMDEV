\subsection{The Rockflow Program} GeoSys/Rockflow is currently in the
third generation of its development. Development started in 1985 at
the University of Hannover.  At the time, GeoSys/Rockflow was only
known as Rockflow, and written as separate modules in Fortran.  Each
developer produced a stand-alone module.  By 1996 the following
modules were present (\cite{Festschrift:99}, p. 115-118):
Groundwater flow (\cite{Wol:90}), tracer transport (\cite{Kro:91}),
multiphase flow (\cite{Hel:93}), coupling of boundary and finite
elements (\cite{Sha:94}), density-dependent flow
(\cite{RatKolZie:96}).

After that, came the second phase of the code's development, which
was driven by applications in the field of environmental geology and
geothermics (\cite{LegKolZie:96}, \cite{Kol:97}), including projects
like the geothermal hot dry rock research project at
Soultz-sous-For�ts in France and Rosemanowes in the UK
(\cite{KolCla:98}).

>From 1996, the third phase of code development was characterized by
radical changes: The program was rewritten in ANSI-C to enable the
use of dynamic data structures and object-oriented programming.
Another significant change in the structure of the program was, that
whereas developers still wrote their own model, an effort was made
to couple these models, as demanded by applications. Coupling models
meant the introduction of version management and a more unified data
structure. The coupling of the models allows Rockflow to start
playing a role in the simulation of coupled processes: Models for
reactive transport (\cite{Hab:01}), adaptive methods and groundwater
flow (\cite{Kai:01}), multiphase flow (\cite{Tho:01}), gas transport
and heat transport (\cite{Kol:96}) emerge. Acompanying process
modeling, adaptive mesh methods are developed (\cite{Bar:97},
Schulze-Ruhfus).

In parallel to development of numerical methods, part of the
research was invested in structural modeling methods. Methods for
geometric descriptions of porous and fractured media evolve
(\cite{Rot:01}, Kasper).  These mesh generation and meshing methods
are essential to the adaptive mesh process modeling methods.  Sylvia
Moenickes worked specifically on mesh generation for fractured
porous media (\cite{Moe:04}). In this field, a good cooperation
exists with Prof. Takeo Taniguchi from Okayama University, Japan.

The coupling of fluid flow and mechanical processes is of particular
interest in soil mechanics.  In 2000, first approaches in the
development of RockFlow were made in order to handle the
consolidation problem (\cite{KolKohTho:00}).  Swelling of bentonite
material began to be an important field of research
(\cite{KohKaiKolZie:02}, \cite{KohKaiKolZie:02}).

The ongoing code developments at the University of Hannover focus on
coupled thermal, hydraulic and mechanical processes. Starting from
linear material models and integrating existing and well validated
flow and transport modules,  the treatment of complex simulations in
combination with a user-friendly control system are the aim of the
current research work (Kohlmeier et al. [2003a]).

Since 2001, Rockflow is also developed at the University of
T\"{u}bingen, in the Center for Applied Geoscience (ZAG).  The
group, headed by Olaf Kolditz grew rapidly to encompass various
fields of research. The current developments of the group can be
split into three categories: Conceptual model development, numerical
model development and software development. In the first category,
work has been undertaken for geotechnical applications
(\cite{KolDej:03}, \cite{WanKol:03}, \cite{XieKolTriSch:03}),
regional groundwater modeling (\cite{BeiKol:03}), groundwater
remediation (\cite{BauKol:03}, \cite{BauXieKol:04},
\cite{Annual:03}), and geothermal reservoir modeling
(\cite{McDKol:03}). In terms of numerical model development, the
finite element library was extended, the code was re-organized to
benefit from further object-orientation and to allow an easier
switching between process couplings. This will be elaborated in the
code section of this paper. In terms of software development,
activities can be summarized as follows: reorganization of RockFlow
into GeoSys: GEOLib, MSHLib, FEMLib, creation and encapsulation of
process-oriented objects (PCS), code parallelization (in cooperation
with the HPC Center Stuttgart), development of GUI (Multi-View, 3-D
graphics).
